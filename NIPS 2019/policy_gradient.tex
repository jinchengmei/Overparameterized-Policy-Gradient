\section{Over-parametrized Policy Gradient}
\label{sec:policy_gradient}

Our first algorithm follows the widely used policy gradient method, as shown in \cref{alg:policy_gradient_uniform_exploration}.
The algorithm updates the policy neural network by doing policy gradient ascent, but with the empirically estimated reward as its objective. 
Note that in contrast a traditional policy gradient method would rather use the current reward as an estimate of the expected reward for its learning objective.
To balance between exploration and exploitation, the agent also has a uniform exploration component in its policy. 
In particular, we set every action with probability at least $t^{ \beta - \frac{1}{3}} \cdot \frac{1}{h}$ to be explored. 


\begin{algorithm}[t]
   \caption{Policy Gradient with Uniform Exploration}
\label{alg:policy_gradient_uniform_exploration}
\begin{algorithmic}
   \STATE {\bfseries Input:} State feature $\rvs$, learning rate $\eta > 0$, $\beta > 0$.
   \STATE $\rvw_r(0) \sim \gN\left( 0, \sigma^2 \cdot \rmI \right)$, $\forall r \in [m]$. $a_{k, r} \sim 
   \unif\left\{-1, +1\right\}$, $\forall k \in [h]$, $\forall r \in [m]$.
   \STATE $\hat{r}_{0}(k) \gets 0$, $n_{0}(k) \gets 0$, $\tilde{\pi}_0(k) \gets \frac{1}{h}$, $\forall k \in [h]$.
   \FOR{$t=0$ {\bfseries to} $T-1$}
   \STATE Sample action $A_{t} \sim \tilde{\rvpi}_{t}\left(\cdot \middle| \rvs \right)$.
   \STATE $\rmW_{t+1} \leftarrow \rmW_t + \eta \cdot \frac{d \rvpi\left(\rmW_t\right)^\top \hat{\rvr}_t}{d \rmW_t}$.
   
   \STATE Take action $A_{t}$. Observe reward $R_{ A_{t}}\left(n_{t}\left(A_t\right) \right)$.
   \STATE $n_{t+1}(k) \gets \left. 
		\begin{cases}
		n_{t}(k) + 1, & \text{if } k = A_t, \\
		n_{t}(k), & \text{otherwise}.
		\end{cases}
		\right. \qquad$ 
   $\hat{r}_{t+1}(k) \gets \left. 
		\begin{cases}
		\frac{n_{t}(k) \cdot \hat{r}_{t}(k) + R_{k}\left(n_{t}(k)\right) }{n_{t+1}(k)}, & \text{if } k = A_t, \\
		\hat{r}_{t}(k), & \text{otherwise}.
		\end{cases}
		\right.$
   \STATE $\tilde{\pi}_{t+1}(k) \gets \left( 1 - \left(t+1\right)^{ \beta - \frac{1}{3}} \right) \cdot \pi\left(\rmW_{t+1}\right)(k) + \left(t+1\right)^{ \beta - \frac{1}{3}} \cdot \frac{1}{h}$, $\forall k \in [h]$.
   \ENDFOR
\end{algorithmic}
\end{algorithm}


Our first main result is a $\tilde{O}(T^{2/3})$ regret for the above policy gradient method, as shown in \cref{thm:policy_gradient_main_result}.

\begin{thm}
\label{thm:policy_gradient_main_result}
    Given policy neural networks as shown in \cref{fig:nn_policy_value}, with number of parameters $m \ge \frac{T^2}{h^2}$, $\eta = \frac{1}{2 h m}$, $\beta = \frac{ \ln{\left(\frac{h}{6}\right) + \ln{\ln{t}} } }{ 3 \ln{t}}$, the expected regret of \cref{alg:policy_gradient_uniform_exploration} satisfies,
\begin{equation*}
\begin{split}
    \sum\limits_{t=0}^{T-1}{ \left( {\rvpi^*}^\top \rvr - \sE \left[ r\left(A_t\right) \right] \right)} \le \frac{6 h}{c} \cdot  T^{\frac{2}{3}} + 5 \cdot T^{\frac{2}{3}} \cdot \left( h \ln{T} \right)^{\frac{1}{3}} + \frac{h}{6} \cdot \ln{ \left( \frac{h}{6} \right) },
\end{split}
\end{equation*}
with probability at least xxxxx. 
\end{thm}
\begin{proof}
According to \cref{alg:policy_gradient_uniform_exploration}, $\tilde{\rvpi}_t$ is used for sampling, which is $\rvpi\left( \rmW_t \right)$ mixed with decaying uniform. Therefore the regret is divided by the linearity of the expectation.
\begin{equation}
\label{eq:total_regret_decomposition}
\begin{split}
    \sum\limits_{t=0}^{T-1}{ \left( {\rvpi^*}^\top \rvr - \sE \left[ r\left(A_t\right) \right] \right) } &\le 1 + \sum\limits_{t=1}^{T-1}{ \left[ \left( 1 - t^{ \beta - \frac{1}{3}} \right) \cdot \left( {\rvpi^*} - \rvpi\left( \rmW_t \right) \right)^\top \rvr + t^{ \beta - \frac{1}{3}} \cdot \left( {\rvpi^*}^\top \rvr - \expectation\limits_{A_t \sim \unif[h]}{ \left[ r\left(A_t\right) \right] } \right) \right]} \\
    &\le 1 + \sum\limits_{t=1}^{T-1}{ \left[ \left( {\rvpi^*} - \rvpi\left( \rmW_t \right) \right)^\top \rvr + t^{ \beta - \frac{1}{3}} \right]} \\
    &\le 1 + \sum\limits_{t=1}^{T-1}{ \left( \frac{ h \ln{t} }{t} \right)^{\frac{1}{3}} } + \sum\limits_{t=1}^{T-1}{ \left( {\rvpi^*} - \rvpi\left( \rmW_t \right) \right)^\top \rvr } \\
    &\le 2 \cdot T^{\frac{2}{3}} \cdot \left( h \ln{T} \right)^{\frac{1}{3}} + \sum\limits_{t=1}^{T-1}{ \left( {\rvpi^*} - \rvpi\left( \rmW_t \right) \right)^\top \rvr },
\end{split}
\end{equation}
where the second inequality comes from $1 - t^{ \beta - \frac{1}{3}} \le 1$, $\forall t \ge 1$, and $\left\| \rvr \right\|_\infty \le 1$.

Denote $\rvpi_t^* \triangleq \argmax\limits_{\rvpi \in \Delta^{h-1}}{\left\{ \rvpi^\top \hat{\rvr}_t\right\}}$. The last term can be decomposed as follows, $\forall t \ge 1$,
\begin{equation}
\label{eq:playing_learning_phase_regret_decomposition}
\begin{split}
    \left( {\rvpi^*} - \rvpi\left( \rmW_t \right) \right)^\top \rvr &= \left( {\rvpi^*} - {\rvpi_t^*} \right)^\top \hat{\rvr}_t + \left( {\rvpi_t^*} - \rvpi\left( \rmW_t \right) \right)^\top \hat{\rvr}_t + \left( {\rvpi^*} - \rvpi\left( \rmW_t \right) \right)^\top \left( \rvr - \hat{\rvr}_t \right) \\
    &\le \left( {\rvpi_t^*} - \rvpi\left( \rmW_t \right) \right)^\top \hat{\rvr}_t  + \left\| \rvpi^* - \rvpi\left( \rmW_t \right) \right\|_1 \cdot \left\| \rvr - \hat{\rvr}_t \right\|_\infty \\
    &\le \left( {\rvpi_t^*} - \rvpi\left( \rmW_t \right) \right)^\top \hat{\rvr}_t + 2 \cdot \left( 2 + \left( h \ln{t} \right) ^\frac{1}{3} \right) \cdot t^{- \frac{1}{3}},
\end{split}
\end{equation}
where the first inequality is by the definition of $\rvpi_t^*$ and  H{\"o}lder's inequality and the second inequality results from \cref{thm:reward_estimation_hoeffding}. Summing up \cref{eq:playing_learning_phase_regret_decomposition} from $1$ to $T-1$, and
combining \cref{eq:total_regret_decomposition} and \cref{thm:dynamic_regret_sublinear},
\begin{equation*}
\begin{split}
    \sum\limits_{t=0}^{T-1}{ \left( {\rvpi^*}^\top \rvr - \sE \left[ r\left(A_t\right) \right] \right) } &\le 2 \cdot T^{\frac{2}{3}} \cdot \left( h \ln{T} \right)^{\frac{1}{3}} + \frac{5 h}{c} \cdot  T^{\frac{2}{3}} + 3 \cdot \left( 2 + \left( h \ln{T} \right) ^\frac{1}{3} \right) \cdot T^{ \frac{2}{3}} \\
    &\le \frac{6 h}{c} \cdot  T^{\frac{2}{3}} + 5 \cdot T^{\frac{2}{3}} \cdot \left( h \ln{T} \right)^{\frac{1}{3}}. \qedhere
\end{split}
\end{equation*}
\end{proof}

\cref{alg:policy_gradient_uniform_exploration} divides the sampling policy $\tilde{\rvpi}_t$ into two parts, i.e., the network policy $\rvpi\left( \rmW_t \right)$ and an uniform policy. Intuitively, when the reward estimation is very inaccurate, updates can probably hurt the neural network policy.  Therefore, an uniform policy is used to guarantee that every action can be sampled for enough many times, which will provide accurate reward estimations for learning. 

\begin{thm}
\label{thm:reward_estimation_hoeffding}
    In \cref{alg:policy_gradient_uniform_exploration}, let $\beta = \frac{ \ln{\left(\frac{h}{6}\right) + \ln{\ln{t}} } }{ 3 \ln{t}}$, we have,
\begin{equation*}
    \left\| \hat{\rvr}_t - \rvr \right\|_\infty \le 2 \cdot t^{- \frac{1}{3}} + t^{\beta - \frac{1}{3}} \le \left( 2 + \left( h \ln{t} \right) ^\frac{1}{3} \right) \cdot t^{- \frac{1}{3}}, \quad \forall t \ge \frac{h}{6} \ln{\left(\frac{h}{6}\right) },
\end{equation*}
with probability at least xxxxx.
\end{thm}
\begin{proof}
    The expected sample number of any action $k$ up to step $t$ can be lower bounded,
\begin{equation*}
\begin{split}
    \expectation{\left[ n_t(k) \right]} &= \sum\limits_{s=0}^{t}{ \tilde{\pi}_t(k)} \ge \sum\limits_{s=1}^{t}{ \tilde{\pi}_t(k)} = \sum\limits_{s=1}^{t}{ \left[ \left( 1 - s^{\beta - \frac{1}{3}} \right) \cdot \pi\left(\rmW_{s}\right)(k) + \frac{s^{\beta - \frac{1}{3}}}{h} \right] } \\
    &\ge \sum\limits_{s=1}^{t}{ \frac{s^{\beta - \frac{1}{3}}}{h} } \ge \frac{t^{\frac{2}{3} + \beta}}{ h  \left(\frac{2}{3} + \beta \right) } \ge \frac{t^{\frac{2}{3} + \beta}}{ h },
\end{split}
\end{equation*}
where the last inequality is by $\beta \le \frac{1}{3}$, $\forall t \ge \frac{h}{6} \ln{\left(\frac{h}{6}\right) }$.
According to the Hoeffding's inequality,
\begin{equation}
\label{eq:loss_estimation_hoeffding}
\begin{split}
    \pr\left\{ \left| \hat{r}_{t}(k) - r(k) \right| > t^{\beta - \frac{1}{3}} \right\} &\le 2 \exp\left\{ - 2 n_{t}(k) \cdot t^{2\beta - \frac{2}{3}} \right\} \le 2 \exp\left\{ - \frac{2}{ h } \cdot t^{3\beta} \right\}.
\end{split}
\end{equation}
Therefore,
\begin{equation*}
\begin{split}
    \left\| \hat{\rvr}_t - \rvr \right\|_\infty &\le \pr\left\{ \left\| \hat{\rvr}_t - \rvr \right\|_\infty \le t^{\beta - \frac{1}{3}} \right\} \cdot t^{\beta - \frac{1}{3}} + \pr\left\{ \left\| \hat{\rvr}_t - \rvr \right\|_\infty > t^{\beta - \frac{1}{3}} \right\} \cdot 1 \\
    &\le 2 \exp\left\{ - \frac{2}{ h } \cdot t^{3\beta} \right\} \cdot 1 + 1 \cdot t^{\beta - \frac{1}{3}},
\end{split}
\end{equation*}
since $\left\| \hat{\rvr}_t - \rvr \right\|_\infty \le 1$, and according to \cref{eq:loss_estimation_hoeffding}. Taking the value of $\beta$ completes the proof.
\end{proof}

The good estimation of the true mean reward will be helpful to train the neural network policy $\rvpi\left( \rmW_t \right)$. However, the optimization theory of over-parameterized neural networks is developed for supervized learning settings \citep{li2018learning,allen2018convergenceB}, where the learning objective functions are fixed. There are two main differences between our results and existing work, (a) in \cref{alg:policy_gradient_uniform_exploration}, the objectives are dynamic with respect to step $t$; (b) the optimal action need enough exploration during learning (explained later on in \cref{subsec:exploration_in_policy_learning}). We first prove the cumulative dynamic expected loss of $\rvpi\left( \rmW_t \right)$ is sublinear. Then we show some intuitions with lemmas, the proofs of which can be found in the appendix.
\begin{thm}
\label{thm:dynamic_regret_sublinear}
    If $m \ge \frac{T^2}{h^2}$, $\eta = \frac{1}{2 h m}$, then the dynamic regret satisfies,
\begin{equation*}
\begin{split}
    \sum\limits_{t=1}^{T-1}{ \left(  {\rvpi_t^*} - \rvpi\left( \rmW_t \right) \right)^\top \hat{\rvr}_t } \le \frac{5 h}{c} \cdot  T^{\frac{2}{3}},
\end{split}
\end{equation*}
with probability at least $1 - \exp\left\{ - \frac{T^2}{16 h} \right\} \ge 1 - \frac{16 h}{T^2}$.
\end{thm}
\begin{proof}
    Denote $\delta_t \triangleq \left( {\rvpi_t^*} - \rvpi\left( \rmW_t \right) \right)^\top \hat{\rvr}_t$. Decompose $\delta_t$ as follows,
\begin{equation}
\label{eq:dynamic_regret_decomposition}
\begin{split}
    \delta_t &= {\rvpi_t^*}^\top \left( \hat{\rvr}_t - \hat{\rvr}_{t-1}\right) + \left( {\rvpi_t^*} - {\rvpi_{t-1}^*} \right)^\top \hat{\rvr}_{t-1}  + \left( {\rvpi_{t-1}^*} - \rvpi\left( \rmW_{t-1} \right) \right)^\top \hat{\rvr}_{t-1} \\
    &\quad + \left(  \rvpi\left( \rmW_{t-1} \right) - \rvpi\left( \rmW_t \right) 
    \right)^\top \hat{\rvr}_{t-1} + \rvpi\left( \rmW_t \right)^\top \left( \hat{\rvr}_{t-1} - \hat{\rvr}_t \right).
\end{split}
\end{equation}
We upper bound each term in the right hand side. Firstly,
\begin{equation*}
\begin{split}
    {\rvpi_t^*}^\top \left( \hat{\rvr}_t - \hat{\rvr}_{t-1}\right) &= \frac{\pi_{t}^*\left(A_{t-1}\right)}{n_{t}\left(A_{t-1}\right)} \left[ R_{ A_{t-1}}\left( n_{t}\left(A_{t-1}\right) -1 \right) - \hat{r}_{t-1}\left(A_{t-1}\right) \right] \\
    &\le \frac{1}{n_{t}\left(A_{t-1}\right)} \le \frac{ h }{t^{\frac{2}{3} + \beta}},
\end{split}
\end{equation*}
by $n_{t}\left(A_{t-1}\right) = n_{t-1}\left(A_{t-1}\right) + 1$, and $n_{t}\left(A_{t-1}\right) \ge \frac{t^{\frac{2}{3} + \beta}}{ h }$, $\forall t \ge 1$. By the definition of ${\rvpi_{t-1}^*}$,
\begin{equation*}
    \left( {\rvpi_t^*} - {\rvpi_{t-1}^*} \right)^\top \hat{\rvr}_{t-1} \le 0.
\end{equation*}
Note that according to the definition of $\delta_t$,
\begin{equation*}
    \left( {\rvpi_{t-1}^*} - \rvpi\left( \rmW_{t-1} \right) \right)^\top \hat{\rvr}_{t-1} = \delta_{t-1}.
\end{equation*}
By \cref{lem:empirically_expected_reward_parameter_smoothness}, $\eta = \frac{1}{2 h m}$, and \cref{lem:gradient_lower_bound},
\begin{equation*}
\begin{split}
    &\left( \rvpi\left( \rmW_{t-1} \right) - \rvpi\left( \rmW_t \right) \right)^\top \hat{\rvr}_{t-1} \le - \frac{1}{4 h m} \left\| \frac{d \rvpi\left( \rmW_{t-1} \right)^\top \hat{\rvr}_{t-1}}{d \rmW_{t-1}} \right\|_F^2 \\
    &\quad = - \frac{1}{4 h m} \sum\limits_{r=1}^{m}{ \left\| \frac{d \rvpi\left( \rmW_{t-1} \right)^\top \hat{\rvr}_{t-1}}{d \rvw_r(t-1)} \right\|_2^2 } \le - \frac{c^2}{4 h} \left[ \left( {\rvpi_{t-1}^*} - \rvpi\left( \rmW_{t-1} \right) \right)^\top \hat{\rvr}_{t-1}  \right]^2 = - \frac{c^2}{4 h} \cdot \delta_{t-1}^2.
\end{split}
\end{equation*}
Using similar arguments,
\begin{equation*}
    \rvpi\left( \rmW_t \right)^\top \left( \hat{\rvr}_{t-1} - \hat{\rvr}_t  \right) \le \frac{ h }{t^{\frac{2}{3} + \beta}}.
\end{equation*}
Plugging the above upper bounds into \cref{eq:dynamic_regret_decomposition},
\begin{equation*}
    \delta_t \le \delta_{t-1} - \frac{c^2}{4 h} \cdot \delta_{t-1}^2 + \frac{ 2 h }{t^{\frac{2}{3} + \beta}}.
\end{equation*}
Rearranging and summing up from $1$ to $T-1$,
\begin{equation*}
\begin{split}
    \sum\limits_{t=1}^{T-1}{\delta_{t}^2} = \sum\limits_{t=2}^{T}{\delta_{t-1}^2} \le \frac{4 h}{ c^2} \cdot  \sum\limits_{t=2}^{T} { \left[ \delta_{t-1} - \delta_t + \frac{ 2 h }{t^{\frac{2}{3} + \beta}} \right] } \le \frac{25 h^2}{ c^2} \cdot T^{\frac{1}{3}}.
\end{split}
\end{equation*}
By the Root-Mean Square-Arithmetic Mean inequality,
\begin{equation*}
\begin{split}
    \sum\limits_{t=1}^{T-1}{\delta_{t}} \le \sqrt{\left(T  - 1 \right) \cdot \sum\limits_{t=1}^{T-1}{\delta_{t}^2}} \le \frac{5 h}{c} \cdot  T^{\frac{2}{3}}.
\end{split}
\end{equation*}
Let $\sigma = \frac{2 \sqrt{2}}{\sqrt{\pi m}}$ and $\tau = \frac{\sigma \sqrt{\pi}}{2 \sqrt{2}} = \frac{1}{\sqrt{m}}$ in \cref{lem:gradient_coupling}. Since $m \ge \frac{T^2}{h^2}$, we have $T \le \frac{\tau}{2 \eta}$ such that \cref{lem:gradient_coupling} holds during all the playing-learning phase. According to \cref{lem:gradient_coupling_in_total}, with probability at least $1 - \exp\left\{ - \frac{m}{8} \left( 1 - \frac{\sqrt{2}\tau}{\sqrt{\pi}\sigma} \right) \right\} = 1 - \exp\left\{ - \frac{m}{16} \right\} \ge 1 - \exp\left\{ - \frac{T^2}{16 h} \right\}$,
\end{proof}



\cref{thm:dynamic_regret_sublinear} relies on two arguments. First, the dynamic expected loss is smooth in the logit space, and small policy gradient updates preserve the signs of ReLU outputs, therefore highly correlates the logit derivative and the policy gradient. Second, by the over-parameterization theory, gradient norm is lower bounded by expected loss around initialization, which means there is no bad local minima near the randomly initialized neural network policy $\rvpi\left( \rmW_0 \right)$.

\begin{lem}
\label{lem:logit_smoothness}
Let $\rvpi(\rvo^\prime)$ and $\rvpi( \rvo)$ be the softmax policies of any logit vectors $\rvo^\prime, \rvo \in \sR^h$, respectively. $\forall \rvr \in \left[ -1, 1\right]^h$,
\begin{equation*}
    \rvpi(\rvo^\prime)^\top \rvr \le \rvpi(\rvo)^\top \rvr + \left\langle \frac{d \rvpi(\rvo)^\top \rvr}{d \rvo}, \rvo^\prime - \rvo \right\rangle + \left\| \rvo^\prime - \rvo \right\|_2^2.
\end{equation*}
\end{lem}
Let $\rvo_{t+1} = \rvo_{t} + \eta \cdot \frac{d \rvpi(\rvo_{t})^\top \hat{\rvr}_t}{d \rvo_{t}}$, and let $\rvr = - \hat{\rvr}_t$, $\eta = \frac{1}{2}$ in \cref{lem:logit_smoothness},
\begin{equation*}
\begin{split}
\small
    \rvpi\left( \rvo_{t+1} \right)^\top \hat{\rvr}_t - \rvpi(\rvo_{t})^\top \hat{\rvr}_t \ge\frac{1}{4} \left\| \frac{d \rvpi(\rvo_{t})^\top \hat{\rvr}_t}{d \rvo_{t}} \right\|_2^2.
\end{split}
\end{equation*}
Note the logit derivative norm is lower bounded by loss,
\begin{equation}
\label{eq:logit_derivative_lower_bound}
\begin{split}
    \left\| \frac{d \rvpi(\rvo_{t})^\top \hat{\rvr}_t}{d \rvo_{t}} \right\|_2 \ge \pi_{t}(\hat{k}_t^*) \cdot \left( \hat{r}_{t}(\hat{k}_t^*) - \rvpi(\rvo_{t})^\top \hat{\rvr}_t \right),
\end{split}
\end{equation}
where $\hat{k}_t^* \triangleq \argmax\limits_{ k \in [h]}{\left\{ \hat{r}_t(k) \right\}}$. Therefore, by \cref{lem:logit_smoothness} and \cref{eq:logit_derivative_lower_bound}, we can derive similar relationship for $\delta_t$ and sublinear regret for update in the logit space. However, in practice, $\rvpi$ is updated in the parameter space. When the parameters are updated from $\rmW^\top \triangleq \left[ \rvw_1, \rvw_2, \dots, \rvw_m \right]$ to ${\rmW^\prime}^\top \triangleq \left[ \rvw_1^\prime, \rvw_2^\prime, \dots, \rvw_m^\prime \right]$,
\begin{equation*}
\begin{split}
    \rvo^\prime - \rvo = \rmA \left[ \sigma \left( \rmW^\prime \rvs \right) - \sigma \left( \rmW \rvs \right) \right].
\end{split}
\end{equation*}
The difficulty is that after updating the parameters, possibly many signs in the ReLU components also change. This can be circumvented by restraining the updates around the initialization \citep{li2018learning}.

\begin{lem}
\label{lem:gradient_coupling}
	Define the pseudo policy gradient at step $t$ as,
\begin{equation*}
\begin{split}
	\frac{d \tilde{\ell}_t}{d \rmW_t} \triangleq \tilde{\rmD} \rmA^\top \rmH\left( \rvpi\left(\rmW_t\right) \right) \hat{\rvr}_t \rvs^\top,
\end{split}
\end{equation*}
where $\tilde{\rmD} \in \sR^{h \times h}$ is a diagonal matrix, and  $\tilde{\rmD}_{r,r} \triangleq \sI\left\{ \rvw_r(0)^\top \rvs > 0 \right\}$, $\forall r \in [m]$. $\rmH\left( \rvpi\left(\rmW_t\right) \right)$ is,
\begin{equation*}
    \rmH\left( \rvpi\left(\rmW_t\right) \right) \triangleq \Delta\left( \rvpi\left(\rmW_t\right) \right) - \rvpi\left(\rmW_t\right) \rvpi\left(\rmW_t\right)^\top.
\end{equation*}
The true policy gradient is,
\begin{equation*}
\begin{split}
    \frac{d \rvpi\left(\rmW_t\right)^\top \hat{\rvr}_t}{d \rmW_t} \triangleq  \rmD(t) \rmA^\top \rmH\left( \rvpi\left(\rmW_t\right) \right) \hat{\rvr}_t \rvs^\top.
\end{split}
\end{equation*}
where $\rmD_{r,r}(t) \triangleq \sI\left\{ \rvw_r(t)^\top \rvs > 0 \right\}$, $\forall r \in [m]$. $\forall \tau > 0$, $\forall r \in [m]$, with probability at least $1 - \frac{\sqrt{2}\tau}{\sqrt{\pi}\sigma}$, $\forall t \le \frac{\tau}{ 2 \eta }$,
\begin{equation*}
	\frac{d\tilde{\ell}_t}{d \rvw_r(t)} = \frac{d \rvpi\left(\rmW_t\right)^\top \hat{\rvr}_t}{d \rvw_r(t)},
\end{equation*}
where $\rvw_r(t)$ is the $r$th row vector of $\rmW_t$.
\end{lem}

\cref{lem:gradient_coupling} implies that for bounded numbers of policy gradient updates, the signs of the ReLUs will not change, i.e., $\sI\left\{ \rvw_r(t)^\top \rvs > 0 \right\} = \sI\left\{ \rvw_r(0)^\top \rvs > 0 \right\}$. Combine \cref{lem:logit_smoothness} with \cref{lem:gradient_coupling}, we have the smoothness property of the surrogate expected loss in the parameter space.
\begin{lem}
\label{lem:empirically_expected_reward_parameter_smoothness}
    $\rmW_{t+1} = \rmW_t + \eta \cdot \frac{d \rvpi\left(\rmW_t\right)^\top \hat{\rvr}_t}{d \rmW_t}$. $\forall t \le \frac{\tau}{ 2 \eta }$,
\begin{equation}
\label{eq:parameter_smoothness}
\begin{split}
    \rvpi\left( \rmW_t \right)^\top \hat{\rvr}_t - \rvpi\left( \rmW_{t+1} \right)^\top \hat{\rvr}_t \le - \left( \eta - h m \eta^2 \right) \cdot \left\| \frac{d \rvpi\left( \rmW_t \right)^\top \hat{\rvr}_t}{d \rmW_t} \right\|_F^2.
\end{split}
\end{equation}
\end{lem}

Now by the key insight of the recent progresses of the over-parameterized neural network optimization theory, with constants probability, the pseudo gradient norm is lower bounded by the objective \citep{li2018learning}. However, unlike the supervised learning, RL has exploration issue, as shown in \cref{subsec:exploration_in_policy_learning}. Our result contains an exploration related term, which is consistent with \cref{eq:logit_derivative_lower_bound}, making guarantees for exploring the optimal action necessary during learning.

\begin{lem}
\label{lem:gradient_lower_bound}
	Denote $\hat{k}_t^* \triangleq \argmax\limits_{k \in [h]}\left\{ \hat{r}_{t}(k) \right\}$, i.e., the optimal action using the estimated reward $ \hat{\rvr}_t$. If $\pi_{t}(\hat{k}_t^*) > c > 0$, with probability $\frac{3}{64} \in \Omega\left( 1 \right)$,
\begin{equation*}
\begin{split}
	\left\| \frac{d\tilde{\ell}_t}{d \rvw_r(t)} \right\|_2 \ge c \cdot \left( \max\limits_{k \in \left[h\right]}\left\{ \hat{r}_{t}(k) \right\} - \rvpi\left( \rmW_t \right)^\top \hat{\rvr}_t \right) .
\end{split}
\end{equation*}
\end{lem}

\cref{lem:gradient_lower_bound} generalizes the over-parameterized neural network optimization theory into the RL settings. By \cref{lem:gradient_lower_bound}, whenever the policy empirically expected reward $\rvpi\left( \rmW_t \right)^\top \hat{\rvr}_t$ small comparing with the largest possible empirical reward $\max\limits_{k \in \left[h\right]}\left\{ \hat{r}_{t}(k) \right\}$, with enough exploration of the suggorate optimal action ($\pi_{t}(\hat{k}_t^*) > c > 0$), with constant probability, the pseudo policy gradient norm will also be large. Therefore, combining \cref{lem:gradient_lower_bound} with \cref{lem:gradient_coupling}, the true policy gradient norm is also large, which is necessary for using \cref{lem:empirically_expected_reward_parameter_smoothness}. Applying all the stated lemmas, the policy surrogate expected loss converges as shown in \cref{thm:dynamic_regret_sublinear}.

\subsection{Exploration of the Optimal Action}
\label{subsec:exploration_in_policy_learning}

Consider the logit derivative of the true mean reward,
\begin{equation}
\label{eq:logit_derivative}
\begin{split}
    \frac{d \rvpi( \rvo)^\top \rvr}{d \rvo} = \left[ \Delta(\rvpi) - \rvpi \rvpi^\top \right] \rvr,
\end{split}
\end{equation}
where $\Delta(\rvpi) \in \sR^{h \times h}$ is a diagonal matrix with $\rvpi$ as its diagonal. Suppose the $k$th action is worth learning, i.e, $r(k) - \rvpi^\top \rvr > 0$ is large, meaning this action has mean reward $r(k)$ much larger than the expected mean reward of the current policy, so the agent should increase its action logit. But if $\pi(k)$ is very close to zero, the increase of the $k$th action logit will be small. Since $\pi(k)$ is small, the $k$th action will be sampled rarely, and with other action logits increasing, $\pi(k)$ will be even smaller, which makes eventually the $k$th action cannot be sampled and learned any more.

\cref{eq:logit_derivative} indicates that to learn the $k$th action, $\pi_{t}(k) > c > 0$ should hold for some constant $c$. In particular, to learn an optimal policy, $\pi_{t}(k^*) > c > 0$ should be guaranteed, $\forall t \ge 0$, where $r(k^*) = \max\limits_{k \in \left[ h \right]}\left\{ r(k) \right\}$.

Consider policy update using \cref{eq:logit_derivative}. $\forall \eta > 0$, $\forall k \in [h]$,
\begin{equation*}
\label{eq:logit_increment_logit_space}
\begin{split}
\small
    o_{t+1}(k) - o_{t}(k) = \eta \cdot \pi_{t}(k) \cdot \left( r(k) - \rvpi(\rvo_{t})^\top \rvr \right),
\end{split}
\end{equation*}
which mean as long as $\pi_{t}(k) > 0$, for any valuable action $k$ with $r(k) -  \rvpi(\rvo_{t})^\top \rvr > 0$, the action logit will increase. While for any bad actions with $r(k) -  \rvpi(\rvo_{t})^\top \rvr < 0$, their logits will decrease.

Consider the uniform policy $\rvpi( \rvo_0)$, with $\pi_{0}(k^*) = \frac{1}{h} \in \Omega(1)$. Also note $r(k^*) - \rvpi( \rvo_0)^\top \rvr$ is larger than any other action $k$, because $k^*$ is the the optimal action. Therefore, the optimal action logit will have the largest positive increment than all the other suboptimal actions. After the softmax transform, the optimal action probability will be larger than its previous value. 
\begin{lem}
\label{lem:optimal_probability_increse_logit_space}
Policy update using \cref{eq:logit_derivative} satisfies, $\forall t \ge 0$,
\begin{equation*}
    \pi_{t+1}(k^*) \ge \pi_{t}(k^*) \in \Omega(1).
\end{equation*}
\end{lem}

\cref{alg:policy_gradient_uniform_exploration} enjoys similar results with \cref{lem:optimal_probability_increse_logit_space}. First, with enough exploration, $\hat{\rvr}_t$ is close to $\rvr$, thus $\rvpi\left(\rmW_t\right)^\top \hat{\rvr}_t$ is close to $\rvpi\left(\rmW_t\right)^\top \rvr$. Second, $\pi_{0}(\hat{k}_t^*) \approx \frac{1}{h} \in \Omega(1)$. Third, since the policy gradient update is around the initialization by \cref{lem:gradient_coupling}, the policy gradient updates behave similarly with the logit derivative updates, which makes the the optimal action logit always large comparing with the suboptimal actions.
\begin{lem}
\label{lem:optimal_probability_increse_parameter_space}
\cref{alg:policy_gradient_uniform_exploration} satisfies, $\forall t \ge 0$,
\begin{equation*}
    \pi_{t}(\hat{k}_t^*) \in \Omega(1).
\end{equation*}
\end{lem}
\cref{lem:optimal_probability_increse_parameter_space} indicates replacing $c$ in \cref{thm:policy_gradient_main_result} will not incur any additional regret dependent on $T$.