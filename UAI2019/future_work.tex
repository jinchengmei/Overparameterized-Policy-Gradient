\section{Future Work}
\label{sec:future_work}

Our results are just at the beginning of theoretically understanding DRL. There are several open problems to our knowledge as follows.
\begin{itemize}
    \item In \cref{thm:policy_gradient_main_result}, the regret is $O\left( T^{\frac{2}{3} } \left(\log{T}\right)^{\frac{1}{3}}\right)$. We believe that the regret can be improved because the total regret in \cref{thm:policy_gradient_main_result} is decomposed into NN learning part and exploration and estimation error part. The former one is $O\left(\log{T}\right)$, while the later one is $T^{\frac{2}{3} } \left(\log{T}\right)^{\frac{1}{3}}$. With more delicately mixing the learning and the exploration, the later one term should be more close to the former term, leading to a better result.
    \item We just consider uniform exploration. Some other popular strategies, like count based exploration \cite{auer2002finite}, EXP3 \citep{seldin2014one}, posterior sampling \citep{agrawal2012analysis} should also be investigated. However, one complication is that learning in a way like \cref{alg:policy_gradient_uniform_exploration}, it is not known whether the logit will converge to the true reward/value, which is necessary to make these explorations work.
    \item In practice, it can always be found that gradient updates travel a long distance rather than around initialization \citep{liu2018deeptracker}. And usually, practical NNs do not have that much overparamerized scales. Any progresses in the DL theory will be helpful for the  improvements in DRL.
    \item We just investigate the basic cumulative reward based policy gradient method. Many other successful RL algorithms including value based and actor critic methods, e.g, DQN \cite{mnih2015human}, A3C \citep{mnih2016asynchronous}, and PCL \citep{nachum2017bridging}, etc, should also be studied under NN function approximations. Also, other policy optimization methods beyong basic policy gradient, such as relative entropy policy search \citep{peters2010relative}, are also worth investigating.
    \item For policy leaning with NN function approximations, there is no known lower bound, although at least the general $\Omega\left(\sqrt{T}\right)$ lower bound holds. We can get some sense from the two parts of the regret in \cref{thm:policy_gradient_main_result}. The NN learning part seems cannot be improved, since this is the best one can obtain under smoothness like properties and gradient upper $\&$ lower bounds. While the other exploration and estimation error part seems still have space to be improved. But whether $\Omega\left(\sqrt{T}\right)$ is achievable is still unknown.
\end{itemize}

\if0
% In the unusual situation where you want a paper to appear in the
% references without citing it in the main text, use \nocite
\nocite{langley00}
\fi