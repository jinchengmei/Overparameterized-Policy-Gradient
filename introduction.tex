\section{Introduction}
\label{introduction}

Deep Reinforcement Learning (DRL) has recently achieved great successes in many fields, e.g, defeating professional players in Go, two player Hold'em, and Atari games, to name a few. However, comparing with its practical performance, the theoretical understanding and explanation of the mechanism behind its success is not enough.

Since DRL combines techniques of Deep Learning (DL) and Reinforcement Learning (RL), to understand DRL, we need to combine findings from both DL and RL sides.

On the RL side, it is well studied that under bandit setting and tabular cases of Markov decision process (MDP) setting, many popular algorithms enjoy favorable theoretical guarantees. In particular, either the algorithm converges asymptotically, or its regret of has finite time sublinear upper bound. However, once it goes beyond the tabular cases, the theoretical guarantee becomes weaker. For example, in linear function approximation, only fixed point property is known for the algorithm. Even worse, when we combine RL methods with non-linear neural networks, such as DQN algorithm, little is known in theory.

On the DL side, empirical achievements are also much more advanced than theoretical results. However, there are still progresses in the expressiveness, optimization, and generalization aspects of DL theory. In particular, very recently, it has been discovered that, in supervised learning (regression and classification) settings, the training loss can be globally optimized (in linear convergence rate for $\ell_2$ loss in regression, and in constant time for classification) by gradient descent (GD) and stochastic gradient descent (SGD) method, given that the number of parameters in hidden layer is quite large, i.e., overparameterization. With some additional structured data distribution assumptions, the convergent training loss can be generalized to testing loss, making the neural network have provable generalization ability. 

In this paper, based on the recent progresses in overparameterized neural network optimization, we make one step forward to theoretically understanding DRL. In particular, we make the following contributions.
\begin{itemize}
    \item We prove that in bandit setting, with enough exploration, the widely-used policy gradient method, with policy net represented by a overparameterized two layer neural network (NN), achieves $\tilde{O}\left( \sqrt{T} \right)$ regret.
    \item We show that in episodic MDP setting, with the same exploration strategy and policy net as above, policy gradient method achieves $\tilde{O}\left( \sqrt{T} \right)$ regret.
    \item In many state dependent bandit and MDP settings, similar results also hold.
\end{itemize}

We would like to point that the above results can be generalized to policy net represented by multi-layered neural networks. Assuming a two layer NN policy is just for simplicity and conciseness. We provide the generalization to multi-layered NN in the appendix for completeness.

To our knowledge, our finding is the first convergence result for popular RL methods (here, policy gradient) with non-linear NN function approximations, which provides theoretical support for DRL methods. Our result is just beginning of understanding many other DRL methods (like DQN, A3C, PCL, etc), with more practical NN function approximations (e.g., less overparameterized), using many other policy optimization techniques (such as mirror descent, and relative entropy policy search).

The rest of the paper is organized as follows. Some proofs are deferred to appendix due to space limit.

\subsection{Notations}

Bold letters refer to vectors, and non-bold letters refer to scalars. For example, $u_{i,r} \in \sR$ is the $r$th component of vector $\rvu_i \in \sR^m$. $n$ is the total number of states, while $m$ is the total number of nodes in each hidden layer. $h$ is the total number of actions can be taken at each state. $\rvone$ means all-one vector, with dimension depends on the context.

Denote $[n] \triangleq \left\{ 1,2, \dots, n \right\}$. $\rvs_i \in \sR^d$, $i \in [n]$ refers to a state. $\rvw_r \in \sR^d$, $r \in [m]$ is a weight vector in the (first) hidden layer. $\rmW \triangleq \left[ \rvw_1, \rvw_2, \dots, \rvw_m \right] \in \sR^{d \times m}$ is the weight matrix of the (first) hidden layer. $u_{i,r} \triangleq \rvw_r^\top \rvs_i$ is the $r$th node value of the (first) hidden layer. $\rva_k \in \sR^m$, $k \in [h]$ is a weight vector in the second layer. In the paper, after initialization $\rva_k \sim \gN(0, \rmI)$, $\rva_k$ will be fixed. This assumption is common in literature, and has been empirically verified that has no impact on the performance of trained neural networks. $o_{i,k} \triangleq \sum\limits_{r=1}^{m}{a_{k,r} \cdot \sigma\left( u_{i,r} \right)}$ is the logit of the $k$th action for state $\rvs_i$, where $\sigma(\cdot) \triangleq \max\left\{ \cdot, 0 \right\}$ is the ReLU activation function. $\pi_{i,k} \triangleq f\left( o_{i,k} \right) \triangleq \frac{\exp\left\{ o_{i,k} \right\}}{\sum\limits_{k^\prime = 1}^{h}{\exp\left\{ o_{i,k^\prime} \right\}}}$ is the probability of choosing action $k$ at state $\rvs_i$, where $f(\cdot)$ is the softmax function. $\rvr_i \in \sR^h$ is the true reward vector at state $\rvs_i$. $\r_i^{\max} \triangleq \max\limits_{k \in [h]}\left\{ r_{i,k} \right\}$ is the maximum reward at state $\rvs_i$. $\rvtilder_i \triangleq \r_i^{\max} \cdot \rvone - \rvr_i$ is the true loss vector at state $\rvs_i$. Given stochastic policies $\rvpi \triangleq \left[ \rvpi_1, \rvpi_2, \dots, \rvpi_n \right] \in \sR^{h \times n}$, the expected loss is defined as,
\begin{equation}
\label{eq:expected_loss}
\begin{split}
    \ell \triangleq \frac{1}{n} \cdot \sum\limits_{i=1}^{n}{ \left( \r_i^{\max} - \rvpi_i^\top \rvr_i \right) } = \frac{1}{n} \cdot \sum\limits_{i=1}^{n}{ \rvpi_{i}^\top \rvtilder_{i} }.
\end{split}
\end{equation}

Without loss of generality, we assume $\rvr_i \in \left[ 0, 1 \right]^h$, $\forall i \in [n]$. Therefore $\rvtilder_i \in \left[ 0, 1 \right]^h$, $\forall i \in [n]$. For simplicity, we also assume $\rvr_i$ is a deterministic vector, $\forall i \in [n]$, while the results also generalize to random reward vectors. Finally, we use $\Delta_i \triangleq \max\limits_{k \in [h]}\left\{ \tilde{r}_{i,k} \middle| \tilde{r}_{i,k} > 0 \right\}$ to denote the reward gap between the optimal arm and the arm with the second largest reward at state $\rvs_i$, $\forall i \in [n]$.