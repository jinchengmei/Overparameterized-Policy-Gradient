\section{Theoretical Analysis}

\subsection{Main Results}
\label{subsec:main_results}

We first present the main results in the bandit setting, as shown in \cref{thm:main_result}, and defer the results of more general settings later.

\begin{thm}
\label{thm:main_result}
    Given a two layer NN policy $\rvpi_i$, with number of parameters $m \in \tilde{\Theta}\left( \frac{\log{T}}{c^4} \right)$, $\eta \in \Theta\left( \frac{c^2}{16 h m \left( \log{\left(4m\right)} \right)^2} \right)$, the expected regret of \cref{alg:policy_gradient_uniform_exploration} satisfies $\sum\limits_{t=0}^{T-1}{ \sE \left[ \tilde{r}_{i, A_t} \right] } \le  \frac{8\sqrt{h}}{c^2} \cdot \tilde{O}\left( \sqrt{T} \right)$.
\end{thm}
\begin{proof}
According to \cref{alg:policy_gradient_uniform_exploration}, the agent uniformly samples and takes actions in the exploring phase, which leads to at most $\sqrt{T}$ regret. At the end of the exploring phase, each action is taken $\frac{\sqrt{T}}{h}$ times in expectation, by \cref{thm:regret_convergence}, 
\begin{equation*}
\begin{split}
    \rvpi_i\left( \rmW(t) \right)^\top \rvtilder_i \le O\left( \frac{n^4}{\eta m c^2 \delta^2 \sqrt{T}} \right),
\end{split}
\end{equation*}
for $t = \sqrt{T}$. Denote $\varepsilon \triangleq \frac{n^4}{\eta m c^2 \delta^2 \sqrt{T}}$. By \cref{subsec:playing_phase}, $\rvpi_i\left( \rmW(t) \right)^\top \rvtilder_i \le \varepsilon$, $\forall t \ge \sqrt{T}$. Therefore the expected regret can be upper bounded by the sum of two parts,
\begin{equation*}
\begin{split}
    \sum\limits_{t=0}^{T-1}{ \sE \left[ \tilde{r}_{i, A_t} \right] } &\le \sum\limits_{t=0}^{\sqrt{T}-1}{ 1 } + \sum\limits_{t=\sqrt{T}}^{T-1}{ \rvpi_i\left( \rmW(t) \right)^\top \rvtilder_i } \\
    &\le \sqrt{T} + \left( T - \sqrt{T} \right) \varepsilon \\
    &\le \frac{8 n^4 \sqrt{h}}{c^2 \delta^2} \cdot \log{\left(4m\right)} \cdot \sqrt{T} \\
    &\le \frac{8 n^4 \sqrt{h}}{c^2 \delta^2} \cdot \tilde{O}\left(\sqrt{T}\right). \qedhere
\end{split}
\end{equation*}
\end{proof}

\cref{alg:policy_gradient_uniform_exploration} divides $T$ steps into two parts. In the first $\sqrt{T}$ exploring phase steps, the agent uniformly samples actions, and learnd the NN policy with expected loss convergent to smaller than $\varepsilon$. Intuitively, this phase is necessary because of: (a) unlike tabular algorithms, i.e., maintaining a scalar value for each arm, the NN policy evolves in a much slower way, due to the usually very small learning rate $\eta$; (b) it is sufficient that the expected loss of the the NN policy is small enough, to make the NN policy have large enough probability for the optimal arm. While in the second playing phase, because of the expected loss of the NN policy is small enough, the gradient norm will be lower bounded by positive constant times expected loss. Therefore the NN policy will keep reducing its expected loss, which means the expected regret will also be reduced. Our detailed analysis also consists of corresponding two parts.

\subsection{Exploring Phase}
\label{subsec:exploring_phase}

As aforementioned, the agent need to learn a good enough NN policy at the end of the exploring phase. The performance of the NN policy can be measured by its expected loss \cref{eq:expected_loss}. And the convergence of the expected loss relies on the following ``smoothness" like property over local change of the NN weights.

\begin{lem}
\label{lem:smoothness}
    Given weights $\rmW \triangleq \left[ \rvw_1, \rvw_2, \dots, \rvw_m \right]$, and $\rmW^\prime \triangleq \left[ \rvw_1^\prime, \rvw_2^\prime, \dots, \rvw_m^\prime \right]$. Denote $\rvpi_i\left( \rmW \right)$ and $\rvpi_i\left( \rmW^\prime \right)$ as the neural network policies parameterized by $\rmW$ and $\rmW^\prime$, respectively. $\forall i \in [n]$,
\begin{equation*}
\begin{split}
    &\rvpi_i\left( \rmW^\prime \right)^\top \rvtilder_i \le \rvpi_i\left( \rmW \right)^\top \rvtilder_i + \left\langle \frac{d \rvpi_i\left( \rmW \right)^\top \rvtilder_i }{d\left( \rmW \right)}, \rmW^\prime - \rmW \right\rangle \\
    &\qquad + 4 \sqrt{m \log{\left(4m\right)}} \cdot \rvpi_i\left( \rmW \right)^\top \rvtilder_i \cdot \left\| \rmW^\prime - \rmW \right\|_F \\
    &\qquad + 4 h m \log{\left(4m\right)} \cdot \left\| \rmW^\prime - \rmW \right\|_F^2.
\end{split}
\end{equation*}
\end{lem}

Usually, combining this smoothness condition with other objective properties like convexity, one can prove convergence results. However, the policy expected loss is non-convex over NN weights. We resort to the recent progresses in the overparameterized NN optimization theory, i.e., the gradient coupling and gradient lower bounds \citep{li2018learning}.

\begin{lem}
\label{lem:gradient_coupling}
	Define the pseudo policy gradient as,
\begin{equation*}
\begin{split}
	\frac{d \tilde{\ell}}{d \rvw_r(t)} &\triangleq \frac{1}{n} \cdot \sum\limits_{i=1}^{n}{ \sum\limits_{k=1}^{h}{ \tilde{r}_{i,k} \cdot \pi_{i,k}(t) \cdot \left( \sum\limits_{k^\prime = 1}^{h}{ a_{k^\prime,r}  \cdot v_{k^\prime,k,i}(t) } \right) } } \\
	&\qquad { { \cdot \sI\left\{ \rvw_r(0)^\top \rvs_i > 0 \right\} \cdot \rvs_i } },
\end{split}
\end{equation*}
where $v_{k^\prime,k,i}(t)$ is defined as,
\begin{equation*}
	v_{k^\prime,k,i}(t) = \begin{cases}
    1 - \pi_{i,k^\prime}(t), & \text{if $k^\prime = k$}, \\
    - \pi_{i,k^\prime}(t), & \text{otherwise}.
  \end{cases}
\end{equation*}
	For any $\tau > 0$, with probability at least $\frac{1}{2} \cdot \left( 1 - \frac{\sqrt{2}n\tau}{\sqrt{\pi}\sigma} \right)$, $\forall t \in O\left(\frac{\tau}{\eta  \sqrt{\log{m}}}\right)$, $\forall r \in [m]$,
\begin{equation}
	\frac{d\ell}{d \rvw_r(t)} = \frac{d \tilde{\ell}}{d \rvw_r(t)},
\end{equation}
where $\frac{d\ell}{d \rvw_r(t)}$ is the true policy gradient,
\begin{equation*}
\begin{split}
    \frac{d\ell}{d \rvw_r(t)} &= \frac{1}{n} \cdot \sum\limits_{i=1}^{n}{ \sum\limits_{k=1}^{h}{  \tilde{r}_{i,k} \cdot \pi_{i,k}(t) \cdot \left( \sum\limits_{k^\prime = 1}^{h}{ a_{k^\prime,r}  \cdot v_{k^\prime,k,i}(t) } \right) }} \\
    &\qquad  {{ \cdot \sI\left\{ \rvw_r(t)^\top \rvs_i > 0 \right\} \cdot \rvs_i } }.
\end{split}
\end{equation*}
\end{lem}

\cref{lem:gradient_coupling} implies that with constant probability, for bounded numbers of policy gradient updates, the signs of the true policy gradient will not change, i.e., $\sI\left\{ \rvw_r(t)^\top \rvs_i > 0 \right\} = \sI\left\{ \rvw_r(0)^\top \rvs_i > 0 \right\}$, thus the true gradient is equal to the pseudo gradient, which is easy to be analyzed and it has nice relationship with the expected loss \cref{lem:gradient_lower_bound}.

The next lemma shows that the true policy gradient norm is upper bounded by the expected loss.

\begin{lem}
\label{lem:gradient_upper_bound}
With probability at least $\frac{1}{2}$,
\begin{equation*}
\begin{split}
	\left\| \frac{d\ell}{d \rvw_r(t)} \right\|_2 \le 2 \sqrt{\log{\left(4m\right)}} \cdot \rvpi_{i}(t)^\top \rvtilder_i.
\end{split}
\end{equation*}
\end{lem}

Now comes the key insight of the recent progresses of the overparameterized NN optimization theory. With constants probability, the pseudo policy gradient norm is lower bounded by the expected loss. However, due to the inherent difference between RL and supervised learning, our result contains an exploration related term, which directly leads to the necessity of the exploring phase in \cref{alg:policy_gradient_uniform_exploration}. 

\begin{lem}
\label{lem:gradient_lower_bound}
	Denote $i^*(t) \triangleq \argmax\limits_{i \in [n]}\left\{\rvpi_i(t)^\top \rvtilder_i \right\}$, $k^*(t) \triangleq \argmax\limits_{k \in [h]}\left\{ r_{i^*(t),k} \right\} = \r_{i^*(t)}^{\max}$, i.e., the optimal arm of state $\rvs_{i^*(t)}$. If $\pi_{i^*(t), k^*(t)}(t) > c_t > 0$, then with probability at least $\Omega\left( \frac{\delta}{n} \right)$,
\begin{equation*}
\begin{split}
	\left\| \frac{d\tilde{\ell}}{d \rvw_r(t)} \right\|_2 \ge \Omega\left( \frac{\delta}{n^2} \right) \cdot c_t \cdot  \rvpi_{i^*(t)}(t)^\top \rvtilder_{i^*(t)}.
\end{split}
\end{equation*}
\end{lem}

\cref{lem:gradient_lower_bound} is a generalization of recent overparameterized NN optimization results into the RL field. By \cref{lem:gradient_lower_bound}, whenever the policy expected loss $\rvpi_{i^*(t)}(t)^\top \rvtilder_{i^*(t)}$ is large, with enough exploration of the optimal action (the term $c_t > 0$ here), with constant probability, the pseudo policy gradient norm will also be large. Therefore, combining \cref{lem:gradient_lower_bound} with \cref{lem:gradient_coupling}, we can get the true policy gradient norm is also large, which is necessary for making progresses to reduce the policy expected loss. Carefully applying all the stated results, we get the policy expected loss convergence result for the exploring phase.

\begin{thm}
\label{thm:regret_convergence}
    Let $\rvpi_i\left( \rmW(t) \right)$ be the NN policy parameterized by the weights $\rmW$. Assume $m \in \tilde{\Theta}\left( \frac{n^{10}}{c^4 \delta^4 \varepsilon^2} \right)$, $\eta \in \Theta\left( \frac{c^2 \delta^2}{16 n^4 h m \left( \log{\left(4m\right)} \right)^2} \right)$, after $t \in O\left( \frac{n^4}{\eta m c^2 \delta^2 \varepsilon} \right)$ policy gradient descent iterations, $\rvpi_i\left( \rmW(t) \right)^\top \rvtilder_i \le \varepsilon$.
\end{thm}
\begin{proof}
    By \cref{lem:gradient_coupling}, let $\tau = \frac{\sigma}{n}$, there are $\Omega\left( m \right)$ of $\rvw_r(t)$ such that $\left\| \frac{d\ell}{d \rvw_r(t)} \right\|_2 = \left\| \frac{d\tilde{\ell}}{d \rvw_r(t)} \right\|_2$, $\forall t \in O\left( \frac{\sigma}{\eta n \sqrt{\log{m}}} \right)$. Let $\rmW(t+1) = \rmW(t) - \eta \cdot \frac{d \ell}{d \rmW(t)}$, by \cref{lem:smoothness},
\begin{equation*}
\begin{split}
    &\rvpi_i\left( \rmW(t+1) \right)^\top \rvtilder_i \le \rvpi_i\left( \rmW(t) \right)^\top \rvtilder_i - \eta \cdot \left\| \frac{d \ell}{d \rmW(t)} \right\|_F^2 \\
    &\qquad + 4 \sqrt{m \log{\left(4m\right)}} \cdot \rvpi_i\left( \rmW \right)^\top \rvtilder_i \cdot \eta \cdot \left\| \frac{d \ell}{d \rmW(t)} \right\|_F \\
    &\qquad + 4 h m \log{\left(4m\right)} \cdot \eta^2 \left\| \frac{d \ell}{d \rmW(t)} \right\|_F^2 \\
    &\le \rvpi_i\left( \rmW(t) \right)^\top \rvtilder_i - \eta \cdot \sum\limits_{r=1}^{m}{ \left\| \frac{d\ell}{d \rvw_r(t)} \right\|_2^2 } \\
    &\qquad + 4 \sqrt{m \log{\left(4m\right)}} \cdot \rvpi_i\left( \rmW \right)^\top \rvtilder_i \cdot \eta \cdot \sum\limits_{r=1}^{m}{ \left\| \frac{d\ell}{d \rvw_r(t)} \right\|_2 } \\
    &\qquad + 4 h m \log{\left(4m\right)} \cdot \eta^2 \cdot \sum\limits_{r=1}^{m}{ \left\| \frac{d\ell}{d \rvw_r(t)} \right\|_2^2 } \\
    &\le \rvpi_i\left( \rmW(t) \right)^\top \rvtilder_i - \left( \rvpi_i\left( \rmW(t) \right)^\top \rvtilder_i \right)^2 \\
    &\cdot \left[ \frac{\eta m c^2 \delta^2}{n^4} - 8 \eta m \sqrt{m} \log{\left(4m\right)} - 16 \eta^2 h m^2 \left( \log{\left(4m\right)} \right)^2 \right] \\
    &= \rvpi_i\left( \rmW(t) \right)^\top \rvtilder_i - \left( \rvpi_i\left( \rmW(t) \right)^\top \rvtilder_i \right)^2 \cdot \Omega\left( \frac{\eta m c^2 \delta^2}{n^4} \right).
\end{split}
\end{equation*}
Divide $\left( \rvpi_i\left( \rmW(t+1) \right)^\top \rvtilder_i\right) \cdot \left( \rvpi_i\left( \rmW(t) \right)^\top \rvtilder_i \right)$,
\begin{equation*}
\begin{split}
    &\frac{1}{\rvpi_i\left( \rmW(t+1) \right)^\top \rvtilder_i} - \frac{1}{\rvpi_i\left( \rmW(t) \right)^\top \rvtilder_i} \ge \\
    &\frac{\rvpi_i\left( \rmW(t) \right)^\top \rvtilder_i}{\rvpi_i\left( \rmW(t+1) \right)^\top \rvtilder_i} \cdot \Omega\left( \frac{\eta m c^2 \delta^2}{n^4} \right) \ge \Omega\left( \frac{\eta m c^2 \delta^2}{n^4} \right).
\end{split}
\end{equation*}
Sum up the inequality from $0$ to $t$,
\begin{equation*}
\begin{split}
    \frac{1}{\rvpi_i\left( \rmW(t) \right)^\top \rvtilder_i} \ge \Omega\left( \frac{\eta m c^2 \delta^2}{n^4} \right) \cdot t.
\end{split}
\end{equation*}
After $t \in O\left( \frac{n^4}{\eta m c^2 \delta^2 \varepsilon} \right)$ iterations, $\rvpi_i\left( \rmW(t) \right)^\top \rvtilder_i \le \varepsilon$. Let $\frac{n^4}{\eta m c^2 \delta^2 \varepsilon} \le \frac{\sigma}{n \eta \sqrt{\log{m}}} = \frac{1}{n \eta \sqrt{m \log{m}}}$, we have $m \ge \frac{n^{10}}{c^4 \delta^4 \varepsilon^2}\log{\left( \frac{n^{10}}{c^4 \delta^4 \varepsilon^2} \right)}$.
\end{proof}

\subsection{Playing Phase}
\label{subsec:playing_phase}

In this phase, we use the learned NN policy to sample and take actions. It is important that  

Fortunately, because of the enough exploration phase, we have the exploration item $c$ large enough.

\begin{lem}
    $c > 1 - \frac{1}{\Delta \sqrt{T}}$.
\end{lem}

With this result and \cref{subsec:exploring_phase}, we can prove that the objective progress is always positive and therefore the policy regret will be always less than $\varepsilon$ in the last $T - \sqrt{T}$ time steps. That is why we can safely use the NN policy to play without worrying lack of exploration issue.

\section{Generalizations}

\subsection{State Dependent Bandits}

\subsection{MDPs}

\subsection{Multi-Layered NN Policies}

\section{Related Work}




\if0
% In the unusual situation where you want a paper to appear in the
% references without citing it in the main text, use \nocite
\nocite{langley00}
\fi

\if0
\begin{table}[t]
\caption{Classification accuracies for naive Bayes and flexible
Bayes on various data sets.}
\label{sample-table}
\vskip 0.15in
\begin{center}
\begin{small}
\begin{sc}
\begin{tabular}{lcccr}
\toprule
Data set & Naive & Flexible & Better? \\
\midrule
Breast    & 95.9$\pm$ 0.2& 96.7$\pm$ 0.2& $\surd$ \\
Cleveland & 83.3$\pm$ 0.6& 80.0$\pm$ 0.6& $\times$\\
Glass2    & 61.9$\pm$ 1.4& 83.8$\pm$ 0.7& $\surd$ \\
Credit    & 74.8$\pm$ 0.5& 78.3$\pm$ 0.6&         \\
Horse     & 73.3$\pm$ 0.9& 69.7$\pm$ 1.0& $\times$\\
Meta      & 67.1$\pm$ 0.6& 76.5$\pm$ 0.5& $\surd$ \\
Pima      & 75.1$\pm$ 0.6& 73.9$\pm$ 0.5&         \\
Vehicle   & 44.9$\pm$ 0.6& 61.5$\pm$ 0.4& $\surd$ \\
\bottomrule
\end{tabular}
\end{sc}
\end{small}
\end{center}
\vskip -0.1in
\end{table}
\fi