\section{Theoretical Analysis}
\label{sec:theoretical_analysis}

\subsection{Main Results}
\label{subsec:main_results}

We first present the main results in the bandit settings, as shown in \cref{thm:main_result}, and then show the detailed proof ideas and  intuitions.

\begin{thm}
\label{thm:main_result}
    Given a two layer NN policy $\rvpi_i$, with number of parameters $m \in \Theta\left( \frac{n^{10}}{c_t^4 \delta^4 \varepsilon^2} \right)$, $\eta \in \Theta\left( \frac{c_t^2 \delta^2}{8 n^4 h m} \right)$, the expected regret of \cref{alg:policy_gradient_uniform_exploration} satisfies $\sum\limits_{t=0}^{T-1}{ \sE \left[ \tilde{r}_{i, A_t} \right] } \le  \frac{8 n^4 \sqrt{h}}{c^2 \delta^2} \cdot T^{\frac{2}{3} + \beta }$, $\forall \beta > 0$.
\end{thm}
\begin{proof}
According to \cref{alg:policy_gradient_uniform_exploration}, $T$ rounds are divided into two parts, where the uniform policy and $\rvpi_i\left( \rmW(t) \right)$ are used to sample and take actions, respectively. Therefore the regret consists of two corresponding parts, i.e., $\sum\limits_{t=0}^{T-1}{ \sE \left[ \tilde{r}_{i, A_t} \right] } =$
\begin{equation*}
\begin{split}
     &\sum\limits_{t=0}^{T^{\frac{2}{3} + \beta}}{ \expectation\limits_{A_t \sim \gU\left[h\right]}{ \left[ \tilde{r}_{i, A_t} \right] }} + \sum\limits_{t=T^{\frac{2}{3} + \beta} + 1}^{T-1}{ \rvpi_i\left( \rmW(t) \right)^\top \rvtilder_i } \\
     &\le \small{\sum\limits_{t=0}^{T^{\frac{2}{3} + \beta}}{ 1 } + \sum\limits_{t=0}^{T-1}{ \rvpi_i\left( \rmW(t) \right)^\top \hat{\rvtilder}_i } + \sum\limits_{t=0}^{T-1}{ \rvpi_i\left( \rmW(t) \right)^\top \left( \rvtilder_i -  \hat{\rvtilder}_i \right) } } \\
     &\le T^{\frac{2}{3} + \beta } + \sum\limits_{t=0}^{T-1}{ \frac{1}{t} } + \sum\limits_{t=0}^{T-1}{ \left\| \rvtilder_i -  \hat{\rvtilder}_i \right\|_\infty },
\end{split}
\end{equation*}
where the first equation is by \cref{eq:vanilla_policy_gradient_expected_regret}, the first inequality is by $\rvtilder_i \in \left[ 0, 1\right]^h$, and the second inequality is according to \cref{thm:regret_convergence} and H{\"o}lder's inequality. The last inequality is by Theorem X.
\end{proof}

\cref{alg:policy_gradient_uniform_exploration} divides $T$ steps into two parts. In the first $\sqrt{T}$ exploring phase steps, the agent uniformly samples actions, and learnd the NN policy with expected loss convergent to smaller than $\varepsilon$. Intuitively, this phase is necessary because of: (a) unlike tabular algorithms, i.e., maintaining a scalar value for each arm, the NN policy evolves in a much slower way, due to the usually very small learning rate $\eta$; (b) it is sufficient that the expected loss of the the NN policy is small enough, to make the NN policy have large enough probability for the optimal arm. While in the second playing phase, because of the expected loss of the NN policy is small enough, the gradient norm will be lower bounded by positive constant times expected loss. Therefore the NN policy will keep reducing its expected loss, which means the expected regret will also be reduced.

\subsection{Exploration in Policy Learning}
\label{subsec:exploration_in_policy_learning}

In \cref{subsec:vanilla_policy_gradient}, it is claimed that the vanilla policy gradient method will suffer lack of exploration. Now we intuitively explain the reason. Consider the derivative of the expected loss with respect to the logits,
\begin{equation}
\label{eq:logit_derivative}
\begin{split}
    \frac{d \rvpi_i\left( \rvo \right)^\top \rvtilder_i}{d \rvo_i} = \left[ \Delta\left( \rvpi_i \right) - \rvpi_i \rvpi_i^\top \right] \rvtilder_i,
\end{split}
\end{equation}
where $\Delta\left( \rvpi_i \right)$ is a diagonal matrix with $\Delta\left( \rvpi_i \right)_{k,k} = \pi_{i,k}$, $\forall k \in [h]$. For the $k$th action, the derivative value is $\pi_{i,k} \cdot \left( \tilde{r}_{i,k} - \rvpi_i^\top \rvtilder_i \right)$. Suppose the $k$th action is worth learning, i.e, $\rvpi_i^\top \rvtilder_i - \tilde{r}_{i,k} > 0$ is large, meaning this action will occur loss $\tilde{r}_{i,k}$ much smaller than the expected loss of the current policy, so the agent should increase its action logit. But if $\pi_{i,k}$ is very close to zero, the increase of the $k$th action logit will be small. Since $\pi_{i,k}$ is small, the $k$th action will be sampled rarely, and with other action logits increasing, $\pi_{i,k}$ will be even smaller, which makes eventually the $k$th action cannot be sampled and learned any more.

\cref{eq:logit_derivative} indicates that to learn the $k$th action, $\pi_{i,k}(t) > c > 0$ should hold for each time step $t$, where $c$ is a constant. In particular, to learn an optimal policy, $\pi_{i,k^*}(t) > c > 0$, where $\tilde{r}_{i,k^*} = 0$.

\subsection{Policy Learning in Logit Space}

We first show that in the logit space, the policy gradient method converges in expected loss. The overparameterization theory briges the gap between the policy gradient in the parameter space and the logit space.

\begin{lem}
\label{lem:logit_smoothness}
Let $\rvpi_i\left( \rvo^\prime \right)$ and $\rvpi_i\left( \rvo \right)$ be softmax policies of logit vectors $\rvo^\prime, \rvo \in \sR^h$, respectively.
\begin{equation*}
\small
    \rvpi_i\left( \rvo^\prime \right)^\top \rvtilder_i \le \rvpi_i\left( \rvo \right)^\top \rvtilder_i + \left\langle \frac{d \rvpi_i\left( \rvo \right)^\top \rvtilder_i}{d \rvo}, \rvo^\prime - \rvo \right\rangle + 2 \left\| \rvo^\prime - \rvo \right\|_2^2.
\end{equation*}
\end{lem}
Let $\rvo(t+1) = \rvo(t) - \eta \cdot \frac{d \rvpi_i\left( \rvo(t) \right)^\top \rvtilder_i}{d \rvo(t)}$, by \cref{eq:logit_derivative},
\begin{equation*}
\begin{split}
\small
    \rvpi_i\left( \rvo(t+1) \right)^\top \rvtilder_i \le \rvpi_i\left( \rvo(t) \right)^\top \rvtilder_i - \left( \eta - 2 \eta^2 \right) \left\| \frac{d \rvpi_i\left( \rvo(t) \right)^\top \rvtilder_i}{d \rvo(t)} \right\|_2^2.
\end{split}
\end{equation*}
Combining with the following from \cref{eq:logit_derivative}, let $\eta = \frac{1}{4}$,
\begin{equation}
\label{eq:logit_derivative_lower_bound}
\begin{split}
    \left\| \frac{d \rvpi_i\left( \rvo \right)^\top \rvtilder_i}{d \rvo_i} \right\|_2 &\ge \left| \pi_{i,k^*} \cdot \left( \tilde{r}_{i,k^*} - \rvpi_i^\top \rvtilder_i \right) \right| \\
    &= \pi_{i,k^*} \cdot \rvpi_i^\top \rvtilder_i,
\end{split}
\end{equation}
we have $\rvpi_i\left( \rvo(t) \right)^\top \rvtilder_i \le \frac{8}{c^2 t}$, $\forall t > 0$. It is consistent with \cref{subsec:exploration_in_policy_learning} that small $c$ will lead to long convergent time.

\subsection{Overparameterized NN Policy}
\label{subsec:overparameterized_nn_policy}

In practice, $\rvpi_i$ is updated in the parameter space, i.e., $\rmW$. To use \cref{lem:logit_smoothness}, when the parameters are updated from $\rmW^\top \triangleq \left[ \rvw_1, \rvw_2, \dots, \rvw_m \right]$ to ${\rmW^\prime}^\top \triangleq \left[ \rvw_1^\prime, \rvw_2^\prime, \dots, \rvw_m^\prime \right]$,
\begin{equation*}
\begin{split}
    \rvo_i^\prime - \rvo_i = \rmA \left[ \sigma \left( \rmW^\prime \rvs_i \right) - \sigma \left( \rmW \rvs_i \right) \right].
\end{split}
\end{equation*}
The difficulty is that after updating the parameters, possibly many signs in the ReLU components also change. Overparameterization theory circumvents this by restraining the gradient updates around the initialization \citep{li2018learning}.

\begin{lem}
\label{lem:gradient_coupling}
	Define the pseudo policy gradient as,
\begin{equation*}
\begin{split}
	\frac{d \tilde{\ell}(t)}{d \rmW(t)} \triangleq \tilde{\rmD} \rmA^\top \left[ \Delta\left( \rvpi_i(t) \right) - \rvpi_i(t) \rvpi_i(t)^\top \right] \rvtilder_i \rvs^\top,
\end{split}
\end{equation*}
where $\tilde{\rmD}_{k,k} \triangleq \sI\left\{ \rvw_r(0)^\top \rvs_i > 0 \right\}$, $\forall k \in [m]$, is a diagonal matrix. Note the true policy gradient is,
\begin{equation*}
\begin{split}
    \frac{d \rvpi_i(t)^\top \rvtilder_i}{d \rmW(t)} \triangleq \rmD(t) \rmA^\top \left[ \Delta\left( \rvpi_i(t) \right) - \rvpi_i(t) \rvpi_i(t)^\top \right] \rvtilder_i \rvs^\top.
\end{split}
\end{equation*}
where $\rmD(t)_{k,k} \triangleq \sI\left\{ \rvw_r(t)^\top \rvs_i > 0 \right\}$, $\forall k \in [m]$. For any $\tau > 0$, with probability at least $1 - \frac{\sqrt{2}n\tau}{\sqrt{\pi}\sigma}$, $\forall t \in O\left(\frac{\tau}{\eta}\right)$, $\forall r \in [m]$,
\begin{equation*}
	\frac{d\tilde{\ell}(t)}{d \rvw_r(t)} = \frac{d \rvpi_i(t)^\top \rvtilder_i}{d \rvw_r(t)},
\end{equation*}
where $\rvw_r(t)$ is the $r$th row vector of $\rmW(t)$.
\end{lem}

\cref{lem:gradient_coupling} implies that with constant probability, for bounded numbers of policy gradient updates, the signs of the true policy gradient will not change, i.e., $\sI\left\{ \rvw_r(t)^\top \rvs_i > 0 \right\} = \sI\left\{ \rvw_r(0)^\top \rvs_i > 0 \right\}$, thus combining with \cref{lem:logit_smoothness} leads to the following result.
\begin{lem}
\label{lem:parameter_smoothness}
    Let $\rmW(t+1) = \rmW(t) - \eta \cdot \frac{d \rvpi_i(t)^\top \rvtilder_i}{d \rmW(t)}$, where $t \in O\left( \frac{\tau}{\eta}\right)$,
\begin{equation}
\label{eq:parameter_smoothness}
\begin{split}
\small
    &\rvpi_i\left( \rmW(t+1) \right)^\top \rvtilder_i \le \rvpi_i\left( \rmW(t) \right)^\top \rvtilder_i \\
    &- \eta \cdot \left\| \frac{d \rvpi_i(t)^\top \rvtilder_i}{d \rmW(t)} \right\|_F^2 + 8 h m \eta^2 \left\| \frac{d \rvpi_i(t)^\top \rvtilder_i}{d \rmW(t)} \right\|_F^2.
\end{split}
\end{equation}
\end{lem}

To use \cref{lem:parameter_smoothness} we need upper bound the last term in \cref{eq:parameter_smoothness} by the policy expected loss.
\begin{lem}
\label{lem:gradient_upper_bound}
$\forall r \in [m]$, $\forall t > 0$,
\begin{equation*}
\begin{split}
	\left\| \frac{d \rvpi_i(t)^\top \rvtilder_i}{d \rvw_r(t)} \right\|_2 \le \rvpi_{i}(t)^\top \rvtilder_i.
\end{split}
\end{equation*}
\end{lem}

Now comes the key insight of the recent progresses of the overparameterized NN optimization theory. With constants probability, the pseudo policy gradient norm is lower bounded by the expected loss \citep{li2018learning}. However, unlike the supervised learning setting, RL has exploration issue, as shown in \cref{subsec:exploration_in_policy_learning}. Our result contains an exploration related term, which is consistent with \cref{eq:logit_derivative_lower_bound}, making the exploring phase in \cref{alg:policy_gradient_uniform_exploration} necessary.

\begin{lem}
\label{lem:gradient_lower_bound}
	$\forall i \in [n]$, define $k_i^* \triangleq \argmin\limits_{k \in [h]}\left\{ \tilde{r}_{i,k} \right\}$, i.e., the optimal action at state $\rvs_{i}$. If $\pi(t)_{i, k_i^*} > c_t > 0$, then with probability at least $\Omega\left( \frac{\delta}{n} \right)$,
\begin{equation*}
\begin{split}
	\left\| \frac{d\tilde{\ell}(t)}{d \rvw_r(t)} \right\|_2 \ge \Omega\left( \frac{\delta}{n^2} \right) \cdot c_t \cdot  \rvpi_{i}(t)^\top \rvtilder_{i}.
\end{split}
\end{equation*}
\end{lem}

\cref{lem:gradient_lower_bound} is a generalizes the overparameterized NN optimization theory into the RL field. By \cref{lem:gradient_lower_bound}, whenever the policy expected loss $\rvpi_{i}(t)^\top \rvtilder_{i}$ is large, with enough exploration of the optimal action ($\pi(t)_{i, k_i^*} > c_t > 0$), with constant probability, the pseudo policy gradient norm will also be large. Therefore, combining \cref{lem:gradient_lower_bound} with \cref{lem:gradient_coupling}, the true policy gradient norm is also large, which is necessary for using \cref{lem:parameter_smoothness}. Carefully applying all the stated results, the policy expected loss converges after the exploring phase.

\begin{thm}
\label{thm:regret_convergence}
    Let $\rvpi_i\left( \rmW \right)$ be the NN policy parameterized by the weights $\rmW$. Assume $m \in \Theta\left( \frac{n^{10}}{c^4 \delta^4 \varepsilon^2} \right)$, $\eta \in \Theta\left( \frac{c_t^2 \delta^2}{8 n^4 h m} \right)$, after $t \in O\left( \frac{n^4}{\eta m c_t^2 \delta^2 \varepsilon} \right)$ policy gradient descent iterations, $\rvpi_i\left( \rmW(t) \right)^\top \rvtilder_i \le \varepsilon$.
\end{thm}
\begin{proof}
    By \cref{lem:gradient_coupling}, let $\tau = \frac{\sigma}{n}$, there are $\Omega\left( m \right)$ of $\rvw_r(t)$ such that $\frac{d\tilde{\ell}(t)}{d \rvw_r(t)} = \frac{d \rvpi_i(t)^\top \rvtilder_i}{d \rvw_r(t)}$, $\forall t \in O\left( \frac{\sigma}{\eta n} \right)$. Let $\rmW(t+1) = \rmW(t) - \eta \cdot \frac{d \rvpi_i(t)^\top \rvtilder_i}{d \rmW(t)}$, by \cref{lem:parameter_smoothness},
\begin{equation*}
\begin{split}
    &\rvpi_i\left( \rmW(t+1) \right)^\top \rvtilder_i \le \rvpi_i\left( \rmW(t) \right)^\top \rvtilder_i \\
    &- \eta \left\| \frac{d \rvpi_i(t)^\top \rvtilder_i}{d \rmW(t)} \right\|_F^2 + 8 h m \eta^2 \left\| \frac{d \rvpi_i(t)^\top \rvtilder_i}{d \rmW(t)} \right\|_F^2 \\
    &\le \rvpi_i\left( \rmW(t) \right)^\top \rvtilder_i - \eta \sum\limits_{r=1}^{m}{ \left\| \frac{d \rvpi_i(t)^\top \rvtilder_i}{d \rvw_r(t)} \right\|_2^2 } + 8 h m \eta^2 \sum\limits_{r=1}^{m}{ \left\| \frac{d \rvpi_i(t)^\top \rvtilder_i}{d \rvw_r(t)} \right\|_2^2 } \\
    &\le \rvpi_i\left( \rmW(t) \right)^\top \rvtilder_i - \left( \rvpi_i\left( \rmW(t) \right)^\top \rvtilder_i \right)^2 \cdot \left[ \frac{\eta m c_t^2 \delta^2}{n^4} - 8 \eta^2 h m^2 \right] \\
    &= \rvpi_i\left( \rmW(t) \right)^\top \rvtilder_i - \left( \rvpi_i\left( \rmW(t) \right)^\top \rvtilder_i \right)^2 \cdot \Omega\left( \frac{\eta m c_t^2 \delta^2}{n^4} \right).
\end{split}
\end{equation*}
Divide $\left( \rvpi_i\left( \rmW(t+1) \right)^\top \rvtilder_i\right) \cdot \left( \rvpi_i\left( \rmW(t) \right)^\top \rvtilder_i \right)$,
\begin{equation*}
\begin{split}
    &\frac{1}{\rvpi_i\left( \rmW(t+1) \right)^\top \rvtilder_i} - \frac{1}{\rvpi_i\left( \rmW(t) \right)^\top \rvtilder_i} \ge \\
    &\frac{\rvpi_i\left( \rmW(t) \right)^\top \rvtilder_i}{\rvpi_i\left( \rmW(t+1) \right)^\top \rvtilder_i} \cdot \Omega\left( \frac{\eta m c_t^2 \delta^2}{n^4} \right) \ge \Omega\left( \frac{\eta m c_t^2 \delta^2}{n^4} \right).
\end{split}
\end{equation*}
Sum up the inequality from $0$ to $t$,
\begin{equation*}
\begin{split}
    \frac{1}{\rvpi_i\left( \rmW(t) \right)^\top \rvtilder_i} \ge \Omega\left( \frac{\eta m c_t^2 \delta^2}{n^4} \right) \cdot t.
\end{split}
\end{equation*}
After $t \in O\left( \frac{n^4}{\eta m c_t^2 \delta^2 \varepsilon} \right)$ iterations, $\rvpi_i\left( \rmW(t) \right)^\top \rvtilder_i \le \varepsilon$. Let $\frac{n^4}{\eta m c_t^2 \delta^2 \varepsilon} \le \frac{\sigma}{n \eta} = \frac{1}{n \eta \sqrt{m}}$, we have $m \ge \frac{n^{10}}{c_t^4 \delta^4 \varepsilon^2}$.
\end{proof}

\subsection{Playing Phase}
\label{subsec:playing_phase}

In this phase, the agent uses the learned NN policy to sample and take actions. If the agent stops learning in this phase, then the expected loss of the maintained NN policy will always be small, therefore the expected regret will be sublinear. On the other hand, if the agent keep learning, the expected loss will decrease after each iteration, thus reduce the expected regret.

\begin{lem}
    $c > 1 - \frac{1}{\Delta \sqrt{T}}$.
\end{lem}

With this result and \cref{subsec:exploring_phase}, we can prove that the objective progress is always positive and therefore the policy regret will be always less than $\varepsilon$ in the last $T - \sqrt{T}$ time steps. That is why we can safely use the NN policy to play without worrying lack of exploration issue.

\section{Generalizations}

\subsection{State Dependent Bandits}

\subsection{MDPs}

\subsection{Multi-Layered NN Policies}

\section{Related Work}




\if0
% In the unusual situation where you want a paper to appear in the
% references without citing it in the main text, use \nocite
\nocite{langley00}
\fi

\if0
\begin{table}[t]
\caption{Classification accuracies for naive Bayes and flexible
Bayes on various data sets.}
\label{sample-table}
\vskip 0.15in
\begin{center}
\begin{small}
\begin{sc}
\begin{tabular}{lcccr}
\toprule
Data set & Naive & Flexible & Better? \\
\midrule
Breast    & 95.9$\pm$ 0.2& 96.7$\pm$ 0.2& $\surd$ \\
Cleveland & 83.3$\pm$ 0.6& 80.0$\pm$ 0.6& $\times$\\
Glass2    & 61.9$\pm$ 1.4& 83.8$\pm$ 0.7& $\surd$ \\
Credit    & 74.8$\pm$ 0.5& 78.3$\pm$ 0.6&         \\
Horse     & 73.3$\pm$ 0.9& 69.7$\pm$ 1.0& $\times$\\
Meta      & 67.1$\pm$ 0.6& 76.5$\pm$ 0.5& $\surd$ \\
Pima      & 75.1$\pm$ 0.6& 73.9$\pm$ 0.5&         \\
Vehicle   & 44.9$\pm$ 0.6& 61.5$\pm$ 0.4& $\surd$ \\
\bottomrule
\end{tabular}
\end{sc}
\end{small}
\end{center}
\vskip -0.1in
\end{table}
\fi