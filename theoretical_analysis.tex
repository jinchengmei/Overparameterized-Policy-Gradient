\section{Theoretical Analysis}
\label{sec:theoretical_analysis}

\subsection{Main Results}
\label{subsec:main_results}

We first present the main results in the bandit settings, as shown in \cref{thm:main_result}, and then show the detailed proof ideas and  intuitions.

\begin{thm}
\label{thm:main_result}
    Given a two layer NN policy $\rvpi_i$, with number of parameters $m \in \Theta\left( \frac{n^{10}}{c_t^4 \delta^4 \varepsilon^2} \right)$, $\eta \in \Theta\left( \frac{c_t^2 \delta^2}{8 n^4 h m} \right)$, the expected regret of \cref{alg:policy_gradient_uniform_exploration} satisfies $\sum\limits_{t=0}^{T-1}{ \sE \left[ \tilde{r}_{i, A_t} \right] } \le  \frac{8 n^4 \sqrt{h}}{c^2 \delta^2} \cdot \log{T} + 2 \cdot T^{\frac{2}{3} + \beta }$, $\forall \beta > 0$.
\end{thm}
\begin{proof}
According to \cref{alg:policy_gradient_uniform_exploration}, $T$ rounds are divided into two parts, where the uniform policy and $\rvpi_i\left( \rmW(t) \right)$ are used to sample and take actions, respectively. Therefore the regret consists of two corresponding parts, i.e., $\sum\limits_{t=0}^{T-1}{ \sE \left[ \tilde{r}_{i, A_t} \right] } =$
\begin{equation*}
\begin{split}
     &\sum\limits_{t=0}^{T^{\frac{2}{3} + \beta}}{ \expectation\limits_{A_t \sim \gU\left[h\right]}{ \left[ \tilde{r}_{i, A_t} \right] }} + \sum\limits_{t=T^{\frac{2}{3} + \beta} + 1}^{T-1}{ \rvpi_i\left( \rmW(t) \right)^\top \rvtilder_i } \\
     &\le \small{\sum\limits_{t=0}^{T^{\frac{2}{3} + \beta}}{ 1 } + \sum\limits_{t=0}^{T-1}{ \rvpi_i\left( \rmW(t) \right)^\top \hat{\rvtilder}_i } + \sum\limits_{t=0}^{T-1}{ \rvpi_i\left( \rmW(t) \right)^\top \left( \rvtilder_i -  \hat{\rvtilder}_i \right) } } \\
     &\le T^{\frac{2}{3} + \beta } + \sum\limits_{t=0}^{T-1}{ \frac{1}{t} } + \sum\limits_{t=0}^{T-1}{ \left\| \hat{\rvtilder}_i - \rvtilder_i \right\|_\infty } \\
     &\le T^{\frac{2}{3} + \beta } + \log{T} - 1 + 2 T \exp\left\{ - \frac{2}{h} \cdot  T^{3\beta} \right\} + T^{\frac{2}{3} + \beta},
\end{split}
\end{equation*}
where the first equation is by \cref{eq:vanilla_policy_gradient_expected_regret}, the first inequality is by $\rvtilder_i \in \left[ 0, 1\right]^h$, and the second inequality is according to \cref{thm:surrogate_expected_loss_convergence} and H{\"o}lder's inequality. The third inequality is by \cref{thm:loss_estimation_hoeffding}. Since $2 T \exp\left\{ - \frac{2}{h} \cdot  T^{3\beta} \right\} \in o(1)$, $\forall \beta > 0$, we obtain the expected regret upper bound.
\end{proof}

\cref{alg:policy_gradient_uniform_exploration} divides $T$ steps into two parts. In the first exploring phase, the agent uniformly samples actions without learning the NN policy. Intuitively, at the beginning, when the loss estimation is very inaccurate, early updating can probably hurt the NN policy.  While in the second playing-learning phase, since the loss estimation is good, the NN policy will keep reducing its surrogate expected loss, which is highly related to its true expected loss, which is identical with the expected regret of this phase.

\subsection{Exploring Phase}
\label{subsec:exploring_phase}

The exploring phase of \cref{alg:policy_gradient_uniform_exploration} provides us good estimations of the true mean loss/reward as follows.
\begin{thm}
\label{thm:loss_estimation_hoeffding}
    After the exploring phase of \cref{alg:policy_gradient_uniform_exploration},
\begin{equation*}
    \left\| \hat{\rvtilder}_i - \rvtilder_i \right\|_\infty \le 2 \exp\left\{ - \frac{2}{h} \cdot  T^{3\beta} \right\} + T^{\beta - \frac{1}{3}}.
\end{equation*}
\end{thm}
\begin{proof}
    The $k$th action is sampled $\frac{T^{\frac{2}{3} + \beta} }{h}$ times, $\forall k \in [h]$. By Hoeffding's inequality, $\forall k \in [h]$,
\begin{equation}
\label{eq:loss_estimation_hoeffding}
\begin{split}
    &\pr\left\{ \left| \hat{\tilde{r}}_{i, k} - \tilde{r}_{i,k} \right| > T^{\beta - \frac{1}{3}} \right\} = \pr\left\{ \left| \hat{r}_{i, k} - r_{i,k} \right| > T^{\beta - \frac{1}{3}} \right\} \\
    &\le 2 \exp\left\{ - 2 \cdot  \frac{T^{\frac{2}{3} + \beta}}{h} \cdot T^{2\beta - \frac{2}{3}} \right\} = 2 \exp\left\{ - \frac{2}{h} \cdot  T^{3\beta} \right\}.
\end{split}
\end{equation}
Denote $k^* = \argmax\limits_{k \in [h]}{ \left| \hat{\tilde{r}}_{i, k} - \tilde{r}_{i,k} \right| }$. Then $\left\| \hat{\rvtilder}_i - \rvtilder_i \right\|_\infty =$
\begin{equation*}
\begin{split}
    &\left| \hat{\tilde{r}}_{i, k^*} - \tilde{r}_{i,k^*} \right| \le 2 \exp\left\{ - \frac{2}{h} \cdot  T^{3\beta} \right\} \cdot 1 + 1 \cdot T^{\beta - \frac{1}{3}},
\end{split}
\end{equation*}
where the last inequality is by \cref{eq:loss_estimation_hoeffding}, $\left| \hat{\tilde{r}}_{i, k^*} - \tilde{r}_{i,k^*} \right| \le 1$, and $\pr\left\{ \left| \hat{\tilde{r}}_{i, k} - \tilde{r}_{i,k} \right| \le T^{\beta - \frac{1}{3}} \right\} \le 1$. 
\end{proof}

\subsection{Playing-Learning Phase}
\label{subsec:playing_learning_phase}

The good estimation of the true mean loss $\hat{\rvtilder}_i$ obtained at the end of the exploring phase will be used to train the NN policy $\rvpi_i\left( \rmW(t) \right)$. We carefully combine the overparameterized NN optimization theory for supervized learning \citep{li2018learning,allen2018convergenceB} with the optimal exploration conditions in RL (explained later on in \cref{subsec:exploration_in_policy_learning}), and show that the ``surrogate" expected loss $\rvpi_i\left( \rmW(t) \right)^\top \hat{\rvtilder}_i$ converges at rate $O\left(\frac{1}{t} \right)$. We first prove the main result, then show some intuitions with lemmas, the proofs of which can be found in the appendix.

\begin{thm}
\label{thm:surrogate_expected_loss_convergence}
    Assume $m \in \Theta\left( \frac{n^{10}}{c^4 \delta^4 \varepsilon^2} \right)$, $\eta \in \Theta\left( \frac{c_t^2 \delta^2}{8 n^4 h m} \right)$, During the playing-learning phase, $\forall t \ge T^{\frac{2}{3} + \beta}$, denote $t^\prime \triangleq t - T^{\frac{2}{3} + \beta} \ge 0$. After $t^\prime = \frac{n^4}{\eta m c_t^2 \delta^2 \varepsilon}$ iterations, $\rvpi_i\left( \rmW(t) \right)^\top \hat{\rvtilder}_i \le \varepsilon$.
\end{thm}
\begin{proof}
    By \cref{lem:gradient_coupling}, let $\tau = \frac{\sigma}{n}$, there are $\Omega\left( m \right)$ of $\rvw_r(t)$ such that $\frac{d\tilde{\ell}(t)}{d \rvw_r(t)} = \frac{d \rvpi_i(t)^\top \rvtilder_i}{d \rvw_r(t)}$, $\forall t \in O\left( \frac{\sigma}{\eta n} \right)$. Let $\rmW(t+1) = \rmW(t) - \eta \cdot \frac{d \rvpi_i(t)^\top \rvtilder_i}{d \rmW(t)}$, by \cref{lem:parameter_smoothness},
\begin{equation*}
\begin{split}
    &\rvpi_i\left( \rmW(t+1) \right)^\top \rvtilder_i \le \rvpi_i\left( \rmW(t) \right)^\top \rvtilder_i \\
    &- \eta \left\| \frac{d \rvpi_i(t)^\top \rvtilder_i}{d \rmW(t)} \right\|_F^2 + 8 h m \eta^2 \left\| \frac{d \rvpi_i(t)^\top \rvtilder_i}{d \rmW(t)} \right\|_F^2 \\
    &\le \rvpi_i\left( \rmW(t) \right)^\top \rvtilder_i - \eta \sum\limits_{r=1}^{m}{ \left\| \frac{d \rvpi_i(t)^\top \rvtilder_i}{d \rvw_r(t)} \right\|_2^2 } \\
    &+ 8 h m \eta^2 \sum\limits_{r=1}^{m}{ \left\| \frac{d \rvpi_i(t)^\top \rvtilder_i}{d \rvw_r(t)} \right\|_2^2 } \\
    &\le \rvpi_i\left( \rmW(t) \right)^\top \rvtilder_i - \left( \rvpi_i\left( \rmW(t) \right)^\top \rvtilder_i \right)^2 \cdot \left[ \frac{\eta m c_t^2 \delta^2}{n^4} - 8 \eta^2 h m^2 \right] \\
    &= \rvpi_i\left( \rmW(t) \right)^\top \rvtilder_i - \left( \rvpi_i\left( \rmW(t) \right)^\top \rvtilder_i \right)^2 \cdot \Omega\left( \frac{\eta m c_t^2 \delta^2}{n^4} \right).
\end{split}
\end{equation*}
Divide $\left( \rvpi_i\left( \rmW(t+1) \right)^\top \rvtilder_i\right) \cdot \left( \rvpi_i\left( \rmW(t) \right)^\top \rvtilder_i \right)$,
\begin{equation*}
\begin{split}
    &\frac{1}{\rvpi_i\left( \rmW(t+1) \right)^\top \rvtilder_i} - \frac{1}{\rvpi_i\left( \rmW(t) \right)^\top \rvtilder_i} \ge \\
    &\frac{\rvpi_i\left( \rmW(t) \right)^\top \rvtilder_i}{\rvpi_i\left( \rmW(t+1) \right)^\top \rvtilder_i} \cdot \Omega\left( \frac{\eta m c_t^2 \delta^2}{n^4} \right) \ge \Omega\left( \frac{\eta m c_t^2 \delta^2}{n^4} \right).
\end{split}
\end{equation*}
Sum up the inequality from $0$ to $t$,
\begin{equation*}
\begin{split}
    \frac{1}{\rvpi_i\left( \rmW(t) \right)^\top \rvtilder_i} \ge \Omega\left( \frac{\eta m c_t^2 \delta^2}{n^4} \right) \cdot t.
\end{split}
\end{equation*}
After $t \in O\left( \frac{n^4}{\eta m c_t^2 \delta^2 \varepsilon} \right)$ iterations, $\rvpi_i\left( \rmW(t) \right)^\top \rvtilder_i \le \varepsilon$. Let $\frac{n^4}{\eta m c_t^2 \delta^2 \varepsilon} \le \frac{\sigma}{n \eta} = \frac{1}{n \eta \sqrt{m}}$, we have $m \ge \frac{n^{10}}{c_t^4 \delta^4 \varepsilon^2}$.
\end{proof}

\cref{thm:surrogate_expected_loss_convergence} relies on two arguments. First, the surrogate expected loss is smooth in the logit space, and small policy gradient updates preserve the signs of ReLU outputs, therefore highly correlates the logit derivative and the policy gradient. Second, by the overparameterization theory, gradient norm is lower bounded by expected loss around initialization, which means there is no bad local minima near the randomly initialized NN policy $\rvpi_i\left( \rmW(0) \right)$.

\begin{lem}
\label{lem:logit_smoothness}
Let $\rvpi_i\left( \rvo^\prime \right)$ and $\rvpi_i\left( \rvo \right)$ be softmax policies of logit vectors $\rvo^\prime, \rvo \in \sR^h$, respectively.
\begin{equation*}
\small
    \rvpi_i\left( \rvo^\prime \right)^\top \hat{\rvtilder}_i \le \rvpi_i\left( \rvo \right)^\top \hat{\rvtilder}_i + \left\langle \frac{d \rvpi_i\left( \rvo \right)^\top \hat{\rvtilder}_i}{d \rvo}, \rvo^\prime - \rvo \right\rangle + 2 \left\| \rvo^\prime - \rvo \right\|_2^2.
\end{equation*}
\end{lem}
Let $\rvo(t+1) = \rvo(t) - \eta \cdot \frac{d \rvpi_i\left( \rvo(t) \right)^\top \hat{\rvtilder}_i}{d \rvo(t)}$, by \cref{lem:logit_smoothness},
\begin{equation*}
\begin{split}
\small
    \rvpi_i\left( \rvo(t+1) \right)^\top \hat{\rvtilder}_i \le \rvpi_i\left( \rvo(t) \right)^\top \hat{\rvtilder}_i - \left( \eta - 2 \eta^2 \right) \left\| \frac{d \rvpi_i\left( \rvo(t) \right)^\top \hat{\rvtilder}_i}{d \rvo(t)} \right\|_2^2.
\end{split}
\end{equation*}
Note the logit derivative norm is lower bounded by loss,
\begin{equation}
\label{eq:logit_derivative_lower_bound}
\begin{split}
    \left\| \frac{d \rvpi_i\left( \rvo \right)^\top \hat{\rvtilder}_i}{d \rvo} \right\|_2 &\ge \left| \pi_{i,\hat{k}_i^*} \cdot \left( \hat{\tilde{r}}_{i,\hat{k}_i^*} - \rvpi_i\left( \rvo \right)^\top \hat{\rvtilder}_i \right) \right| \\
    &= \pi_{i,\hat{k}_i^*} \cdot \rvpi_i\left( \rvo \right)^\top \hat{\rvtilder}_i,
\end{split}
\end{equation}
where $\hat{k}_i^* \triangleq \argmax\limits_{ k \in [h]}{\left\{ \hat{r}_{i, k} \right\}}$, thus $\hat{\tilde{r}}_{i,\hat{k}_i^*} = \max\limits_{k \in [h]}{\left\{ \hat{r}_{i, k} \right\}} - \hat{r}_{i, \hat{k}_i^*} = 0$. Let $\eta = \frac{1}{4}$, by \cref{lem:logit_smoothness} and \cref{eq:logit_derivative_lower_bound}, we have $\rvpi_i\left( \rvo(t) \right)^\top \hat{\rvtilder}_i \le \frac{8}{c^2 t}$, $\forall t > 0$, where $c \triangleq \min\limits_{t \ge 0}{ \left\{ \pi(t)_{i, \hat{k}_i^*}\right\}}  > 0$ because $\rvpi_i\left( \rvo \right)$ is softmax transform of $\rvo$. It is intuitively reasonable that small $c$ means rarely exploring the optimal action $\hat{k}_i^*$, which will lead to long convergent time.

However, \cref{lem:logit_smoothness} is not good enouth because in practice, $\rvpi_i$ is updated in the parameter space rather than the logit space. When the parameters are updated from $\rmW^\top \triangleq \left[ \rvw_1, \rvw_2, \dots, \rvw_m \right]$ to ${\rmW^\prime}^\top \triangleq \left[ \rvw_1^\prime, \rvw_2^\prime, \dots, \rvw_m^\prime \right]$,
\begin{equation*}
\begin{split}
    \rvo^\prime - \rvo = \rmA \left[ \sigma \left( \rmW^\prime \rvs_i \right) - \sigma \left( \rmW \rvs_i \right) \right].
\end{split}
\end{equation*}
The difficulty is that after updating the parameters, possibly many signs in the ReLU components also change. This can be circumvented by restraining the updates around the initialization \citep{li2018learning}.

\begin{lem}
\label{lem:gradient_coupling}
	Define the pseudo policy gradient as,
\begin{equation*}
\begin{split}
	\frac{d \tilde{\ell}(t)}{d \rmW(t)} \triangleq \tilde{\rmD} \rmA^\top \left[ \Delta\left( \rvpi_i(t) \right) - \rvpi_i(t) \rvpi_i(t)^\top \right] \rvtilder_i \rvs^\top,
\end{split}
\end{equation*}
where $\tilde{\rmD}_{k,k} \triangleq \sI\left\{ \rvw_r(0)^\top \rvs_i > 0 \right\}$, $\forall k \in [m]$, is a diagonal matrix. Note the true policy gradient is,
\begin{equation*}
\begin{split}
    \frac{d \rvpi_i(t)^\top \rvtilder_i}{d \rmW(t)} \triangleq \rmD(t) \rmA^\top \left[ \Delta\left( \rvpi_i(t) \right) - \rvpi_i(t) \rvpi_i(t)^\top \right] \rvtilder_i \rvs^\top.
\end{split}
\end{equation*}
where $\rmD(t)_{k,k} \triangleq \sI\left\{ \rvw_r(t)^\top \rvs_i > 0 \right\}$, $\forall k \in [m]$. For any $\tau > 0$, with probability at least $1 - \frac{\sqrt{2}n\tau}{\sqrt{\pi}\sigma}$, $\forall t \in O\left(\frac{\tau}{\eta}\right)$, $\forall r \in [m]$,
\begin{equation*}
	\frac{d\tilde{\ell}(t)}{d \rvw_r(t)} = \frac{d \rvpi_i(t)^\top \rvtilder_i}{d \rvw_r(t)},
\end{equation*}
where $\rvw_r(t)$ is the $r$th row vector of $\rmW(t)$.
\end{lem}

\cref{lem:gradient_coupling} implies that for bounded numbers of policy gradient updates, the signs of the ReLUs will not change, i.e., $\sI\left\{ \rvw_r(t)^\top \rvs_i > 0 \right\} = \sI\left\{ \rvw_r(0)^\top \rvs_i > 0 \right\}$. Combine \cref{lem:logit_smoothness} with \cref{lem:gradient_coupling}, we have the smoothness property of the surrogate expected loss in the parameter space.
\begin{lem}
\label{lem:parameter_smoothness}
    Let $\rmW(t+1) = \rmW(t) - \eta \cdot \frac{d \rvpi_i(t)^\top \rvtilder_i}{d \rmW(t)}$, where $t \in O\left( \frac{\tau}{\eta}\right)$,
\begin{equation}
\label{eq:parameter_smoothness}
\begin{split}
\small
    &\rvpi_i\left( \rmW(t+1) \right)^\top \rvtilder_i \le \rvpi_i\left( \rmW(t) \right)^\top \rvtilder_i \\
    &- \eta \cdot \left\| \frac{d \rvpi_i(t)^\top \rvtilder_i}{d \rmW(t)} \right\|_F^2 + 8 h m \eta^2 \left\| \frac{d \rvpi_i(t)^\top \rvtilder_i}{d \rmW(t)} \right\|_F^2.
\end{split}
\end{equation}
\end{lem}

To use \cref{lem:parameter_smoothness} we need upper bound the last term in \cref{eq:parameter_smoothness} by the policy expected loss.
\begin{lem}
\label{lem:gradient_upper_bound}
$\forall r \in [m]$, $\forall t > 0$,
\begin{equation*}
\begin{split}
	\left\| \frac{d \rvpi_i(t)^\top \rvtilder_i}{d \rvw_r(t)} \right\|_2 \le \rvpi_{i}(t)^\top \rvtilder_i.
\end{split}
\end{equation*}
\end{lem}

Now comes the key insight of the recent progresses of the overparameterized NN optimization theory. With constants probability, the pseudo policy gradient norm is lower bounded by the expected loss \citep{li2018learning}. However, unlike the supervised learning setting, RL has exploration issue, as shown in \cref{subsec:exploration_in_policy_learning}. Our result contains an exploration related term, which is consistent with \cref{eq:logit_derivative_lower_bound}, making the exploring phase in \cref{alg:policy_gradient_uniform_exploration} necessary.

\begin{lem}
\label{lem:gradient_lower_bound}
	$\forall i \in [n]$, define $k_i^* \triangleq \argmin\limits_{k \in [h]}\left\{ \tilde{r}_{i,k} \right\}$, i.e., the optimal action at state $\rvs_{i}$. If $\pi(t)_{i, k_i^*} > c_t > 0$, then with probability at least $\Omega\left( \frac{\delta}{n} \right)$,
\begin{equation*}
\begin{split}
	\left\| \frac{d\tilde{\ell}(t)}{d \rvw_r(t)} \right\|_2 \ge \Omega\left( \frac{\delta}{n^2} \right) \cdot c_t \cdot  \rvpi_{i}(t)^\top \rvtilder_{i}.
\end{split}
\end{equation*}
\end{lem}

\cref{lem:gradient_lower_bound} is a generalizes the overparameterized NN optimization theory into the RL field. By \cref{lem:gradient_lower_bound}, whenever the policy expected loss $\rvpi_{i}(t)^\top \rvtilder_{i}$ is large, with enough exploration of the optimal action ($\pi(t)_{i, k_i^*} > c_t > 0$), with constant probability, the pseudo policy gradient norm will also be large. Therefore, combining \cref{lem:gradient_lower_bound} with \cref{lem:gradient_coupling}, the true policy gradient norm is also large, which is necessary for using \cref{lem:parameter_smoothness}. Carefully applying all the stated results, the policy expected loss converges after the exploring phase.

\subsection{Exploration in Policy Learning}
\label{subsec:exploration_in_policy_learning}

In \cref{subsec:vanilla_policy_gradient}, it is claimed that the vanilla policy gradient method will suffer lack of exploration. Now we intuitively explain the reason. Consider the derivative of the expected loss with respect to the logits,
\begin{equation}
\label{eq:logit_derivative}
\begin{split}
    \frac{d \rvpi_i\left( \rvo \right)^\top \rvtilder_i}{d \rvo_i} = \left[ \Delta\left( \rvpi_i \right) - \rvpi_i \rvpi_i^\top \right] \rvtilder_i,
\end{split}
\end{equation}
where $\Delta\left( \rvpi_i \right)$ is a diagonal matrix with $\Delta\left( \rvpi_i \right)_{k,k} = \pi_{i,k}$, $\forall k \in [h]$. For the $k$th action, the derivative value is $\pi_{i,k} \cdot \left( \tilde{r}_{i,k} - \rvpi_i^\top \rvtilder_i \right)$. Suppose the $k$th action is worth learning, i.e, $\rvpi_i^\top \rvtilder_i - \tilde{r}_{i,k} > 0$ is large, meaning this action will occur loss $\tilde{r}_{i,k}$ much smaller than the expected loss of the current policy, so the agent should increase its action logit. But if $\pi_{i,k}$ is very close to zero, the increase of the $k$th action logit will be small. Since $\pi_{i,k}$ is small, the $k$th action will be sampled rarely, and with other action logits increasing, $\pi_{i,k}$ will be even smaller, which makes eventually the $k$th action cannot be sampled and learned any more.

\cref{eq:logit_derivative} indicates that to learn the $k$th action, $\pi_{i,k}(t) > c > 0$ should hold for each time step $t$, where $c$ is a constant. In particular, to learn an optimal policy, $\pi_{i,k^*}(t) > c > 0$, where $\tilde{r}_{i,k^*} = 0$.

\subsection{Playing Phase}
\label{subsec:playing_phase}

In this phase, the agent uses the learned NN policy to sample and take actions. If the agent stops learning in this phase, then the expected loss of the maintained NN policy will always be small, therefore the expected regret will be sublinear. On the other hand, if the agent keep learning, the expected loss will decrease after each iteration, thus reduce the expected regret.

\begin{lem}
    $c > 1 - \frac{1}{\Delta \sqrt{T}}$.
\end{lem}

With this result and \cref{subsec:exploring_phase}, we can prove that the objective progress is always positive and therefore the policy regret will be always less than $\varepsilon$ in the last $T - \sqrt{T}$ time steps. That is why we can safely use the NN policy to play without worrying lack of exploration issue.

\section{Generalizations}

\subsection{State Dependent Bandits}

\subsection{MDPs}

\subsection{Multi-Layered NN Policies}

\section{Related Work}




\if0
% In the unusual situation where you want a paper to appear in the
% references without citing it in the main text, use \nocite
\nocite{langley00}
\fi

\if0
\begin{table}[t]
\caption{Classification accuracies for naive Bayes and flexible
Bayes on various data sets.}
\label{sample-table}
\vskip 0.15in
\begin{center}
\begin{small}
\begin{sc}
\begin{tabular}{lcccr}
\toprule
Data set & Naive & Flexible & Better? \\
\midrule
Breast    & 95.9$\pm$ 0.2& 96.7$\pm$ 0.2& $\surd$ \\
Cleveland & 83.3$\pm$ 0.6& 80.0$\pm$ 0.6& $\times$\\
Glass2    & 61.9$\pm$ 1.4& 83.8$\pm$ 0.7& $\surd$ \\
Credit    & 74.8$\pm$ 0.5& 78.3$\pm$ 0.6&         \\
Horse     & 73.3$\pm$ 0.9& 69.7$\pm$ 1.0& $\times$\\
Meta      & 67.1$\pm$ 0.6& 76.5$\pm$ 0.5& $\surd$ \\
Pima      & 75.1$\pm$ 0.6& 73.9$\pm$ 0.5&         \\
Vehicle   & 44.9$\pm$ 0.6& 61.5$\pm$ 0.4& $\surd$ \\
\bottomrule
\end{tabular}
\end{sc}
\end{small}
\end{center}
\vskip -0.1in
\end{table}
\fi